Moment resistant steel framed structures stand out with important
performance parameters of structure design such as safety, economy
and aesthetics. Moment resisting steel frames with high ductility
capacity are frequently used in low and medium rise office buildings
in seismic zones.The high ductility characteristics of the moment
frames lie in the simplicity of the system's energy dissipative mechanism
when subjected to horizontal forces.In these systems, especially under
earthquake forces, a large number of yields in the beams are achieved
by inelastic deformation without reducing the strength of the structure.

It is debatable to what extent all column-beam connections in a structure
are fully rigid or ideally pinned in the design. In fact, when the
behavior of the joints used in the moment frame is examined, it is
necessary to use very specific detailing in order for the conections
to show rigid or ideal pinned behavior, whereas the majority of the
joints used are between the two rigid and pinned end boundary states.
These types of connections are classified as semi-rigid connections.

The displacement amplification coefficient is used to obtain the displacement
value of the structure from the elastic displacement value in the
inelastic state.
