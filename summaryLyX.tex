Rubber bearings are used to prevent vibrations in the buildings and
to allow the bearings to be displaced due to thermal expansion in
the bridges. The first use of rubber bearings in order to protect
constructions against earthquake effects, occurred in Pestalozzi primary
school in Skopje, Yugoslavia in 1969. The same horizontal and vertical
stiffness of the rubber supports applied as a single block caused
the bulge to occur due to the weight of the building on the side surfaces.
The French engineer Eugène Freyssinet, who discovered that the axial
loading capacities of the rubber layers were inversely proportional
to their height, suggested strengthening the rubber layers by adding
thin steel plates in the vertical direction. Here the bond between
the layers is provided due to the friction force. Thanks to the vulcanization
method used to ensure that the thin steel plates and rubber layers
adhere to each other, studies and applications of modern seismic isolators
have begun to increase.