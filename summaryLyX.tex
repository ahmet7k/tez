Moment resisting steel frames structures stand out with important
performance parameters of structure design such as safety, economy
and aesthetics. Moment resisting steel frames with high ductility
capacities are frequently used in low and medium rise buildings in
earthquake zones. The high ductility characteristics of the moment
frames in the simplicity of the system's energy damping mechanism
when subjected to horizontal forces. In these systems, especially
under earthquake forces, a great number of yields in the beams are
achieved by inelastic deformation without reducing the strength of
the structure. In such structures, the principle of strong column-weak
beam is applied to ensure that plastic hinges are formed in the beams
before the columns, to increase ductility and delay the collapse of
the structure. This rule results in the selection of larger column
sections than required and uneconomical, extremely safe designs. In
order to solve this problem, especially in the USA, the method is
to design only the outer frames of the buildings as horizontal load
bearing systems. However, one of the main disadvantages of these structures
is that the redistribution of the system is limited(low redundancy).

During the 1994 Northridge and 1995 Kobe earthquakes, heavy and unexpected
damages occurred in rigid connected moment resisting steel frames
structures, and especially in fully welded connections, leading to
a review of the design methods of such structures. Following this
discovery, a consortium of professional associations and researchers
known as the SAC Joint Venture conducted extensive work to identify
the causes of this unexpected behavior and develop recommendations
for building a more robust moment framework. Cracks and damages were
found to be the result of basic joint geometry, lack of control of
basic materials, the use of weld filler metals with low toughness
in the internal structure, divot breakage of column heads, damage
caused by the cover plate, inadequate quality control and other factors.
The resulting research formed the basis for regulatory design requirements
for existing special moment frameworks.

Moment frames can be said to have great displacement properties as
both advantages and disadvantages. It is debatable to what extent
all column-beam joints in a structure are fully rigid or ideally hinged
in the design. In fact, when the behavior of the joints used in the
moment frame is in the strong or weak axis, it is necessary to use
very specific detailing in order to show that the joints exhibit rigid
or ideal hinged behavior, whereas the majority of the joints used
are between two rigid and hinged end boundary states. These connection
types of classified as semi-rigid connections.

Over the past half century, computer aided software tools used by
engineers for structural design have made significant progress. However,
there are still some assumptions made during the modeling and design
of steel structures. Especially for a steel structure designed in
the earthquake zone, the aim of safety and economy is of great importance.
The structural design philosophy in the earthquake regulations underlines
that even in an earthquake with a reasonable probability of occurrence,
absolute safety and damage cannot be ensured. However, it aims to
economically provide a high level of safety in buildings by allowing
some structural and non-structural damage and absorbing inelastic
energy. As a result of this design philosophy, the horizontal design
force prescribed in the regulations is lower than the horizontal force
required to keep the structure within the elastic range. Retention
of the structure in the elastic envelope means that all structural
elements subjected to lateral movement are returned to the initial
state without any permanent deformation and damage, which is to be
maintained, far from practical and rational. 

When the design of moment frames is considered, the main parameter
that governs the design is displacement. This study focuses on the
concept of connection rotation stiffness in the design of steel frame
structures with moment resistance in the literature and earthquake
regulations, and the concept of the reaction coefficient of change
for the actual displacement of regulations to be used in design. The
displacement amplification coefficient is used to obtain the displacement
value of the structure from the elastic displacement value in the
inelastic state. For this purpose, low, medium and high storied rigid
connections moment framed structures were designed for the earthquake
level determined and examined with five different rotational stiffnesses
by linear inelastic static and dynamic analyses. 15 steel frame static
pushover analyses and 18 real-ground motion dynamic time-history analyses
were scaled at the design earthquake level. With these analysis results,
the displacement amplification factor was calculated for each steel
frame.

In low, medium and high-rise frames, $C_{\mathrm{d}}$ coefficients
calculated under average of 4.84 for dynamic analysis and 3.86 for
dynamic analysis are less than the values of 5.50 and 8.00 used by
AISC and TBDY-2018 regulations, respectively. The results show that
the stiffness of the connection in a steel-frame structure changes
the behavior of the structure, thereby changing the displacement amplification
factor, which is an important response coefficient for steel-frame
structures. The difference in the displacement amplification coefficients
shows that the rigidity of the connections used in the structure should
not be ignored if a non-linear evaluation of a moment resisting steel
frame is performed. In addition, the decrease in connection stiffness
has shown that the ductility ratio of the structures is reduced.
