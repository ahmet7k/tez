%% LyX 2.3.3 created this file.  For more info, see http://www.lyx.org/.
%% Do not edit unless you really know what you are doing.
\documentclass[onluarkali,turkce,yukseklisans,bez,deprem]{itutezLyX}
\usepackage[T1]{fontenc}
\usepackage[utf8]{inputenc}
\setcounter{secnumdepth}{3}
\setcounter{tocdepth}{3}
\usepackage{amsmath}
\usepackage{graphicx}

\makeatletter

%%%%%%%%%%%%%%%%%%%%%%%%%%%%%% LyX specific LaTeX commands.
%% Because html converters don't know tabularnewline
\providecommand{\tabularnewline}{\\}

%%%%%%%%%%%%%%%%%%%%%%%%%%%%%% User specified LaTeX commands.
% ---------------------------------------------------------------- %
%                     LATEX Tez Sablonu                            %
%                                                                  %
%                       Surum 1.5.1                                %
% ---------------------------------------------------------------- %

% ---------------------------------------------------------------- %
% ITU Bilisim Enstitusu tarafindan hazirlanmistir.                 %
%                                                                  %
% İTÜ Ayazağa Kampüsü, Bilişim Enstitüsü Binası                    %
% Maslak-34469, İstanbul                                           %
% http://www.be.itu.edu.tr                                         %
% ---------------------------------------------------------------- %

% ---------------------------------------------------------------- %
% documentclass kullanim argumanlari:                              %
% [onluarkali,tekyonlu],[turkce,ingilizce],[yukseklisans,doktora], %
% [bez,karton],[bilisim,fenbilimleri,sosyalbilimler,avrasya,enerji]%
% Ornek kullanim: \documentclass[onluarkali,ingilizce,yukseklisans,%
% karton,fenbilimleri]{itutezLyX}                                     %
% Baska Ornek kullanim: \documentclass[tekyonlu,ingilizce,         %
% yukseklisans,bez,bilisim]{itutezLyX}                                %
% ---------------------------------------------------------------- %
% \documentclass[onluarkali,turkce,yukseklisans,bez,fenbilimleri]{itutezLyX}
% documentclass LyX içerisinde tanımlanmıştır. BE. 2018-10-05
% ---------------------------------------------------------------- %
% Komutlarda buyuk ve kucuk harf ayrimina ozen gostermek           %
% gereklidir. Ornegin argumanin {\"O\u{g}\-ren\-ci Ad{\i}}{SOYADI} %
% yapisinda olmasi demek, adlarda ilk harflerin buyuk ve           %
% digerlerinin kucuk, soyadinda ise butun harflerin buyuk olmasi   %
% gerektigi anlamina gelmektedir.                                  %
% ---------------------------------------------------------------- %

% ---------------------------------------------------------------- %
% Sadece Ad SOYAD yazilmalidir. Unvan yazilmamalidir.              %
% ---------------------------------------------------------------- %
\yazar{Ahmet}{KARABACAK} 
\ogrencino{802161201}

% ---------------------------------------------------------------- %
% Kullanmayacaginiz yapilarin ayirac aralarini bos birakiniz.      %
% Ornek: \unvan{}                                                  %
% ---------------------------------------------------------------- %
\unvan{İnşaat Mühendisi}
 
% ---------------------------------------------------------------- %
% Asagidaki yapilarda birinci oge Turkce, ikinci oge Ingilizce     %
% yapiyi olusturmak icindir.                                       %
% ---------------------------------------------------------------- %

% ---------------------------------------------------------------- %
% Sozcuklerin ilk harfleri buyuk, diger harfler kucuk yazilacak.   %
% ---------------------------------------------------------------- %
\anabilimdali{Deprem M\"uhendisli\u{g}i Anabilim Dal{\i}}{Department of Earthquake Engineering}
\programi{Deprem M\"uhendisli\u{g}i Program{\i}}{Earthquake Engineering Programme}

% ---------------------------------------------------------------- %
% Beyaz cilt hazirlarken bolume verildigi tarih esas alinir.       %
% Bez (mavi-siyah) ciltte ise tezin savunuldugu tarih ay yil       %
% olarak yazilir.                                                  %
% ---------------------------------------------------------------- %
\tarih{KASIM 2019}{NOVEMBER 2019}
\tarihKucuk{Kas{\i}m 2019}{November 2019}

% ---------------------------------------------------------------- %
% Danisman, baslik ve tez tarihi bilgileri icin asagidaki yapi     %
% kullanilmaktadir.                                                %
% ---------------------------------------------------------------- %

% ---------------------------------------------------------------- %
% Danisman bilgisi, beyaz kapakli tezde ilk sayfada var. Bez       %
% (mavi-siyah) ciltte yok. Neden? Cunku ic kapakta gececek.        %                  
% ---------------------------------------------------------------- %
\tezyoneticisi{Doç. Dr. Barlas Özden ÇAĞLAYAN}{\.Istanbul Teknik \"Universitesi}   

\baslik{OPENSEES VE SEISMOSTRUCT PROGRAMLARININ}{DOĞRUSAL OLMAYAN DEPREM ANALİZLERİ İÇİN}{KARŞILAŞTIRILMASI}{}
\title{COMPARISON OF OPENSEES AND SEISMOSTRUCT PROGRAMS}{FOR NONLINEAR EARTHQUAKE ANALYSIS}{}

% ---------------------------------------------------------------- %
% Bu tarih beyaz kapakli ilk tezin bolume verildigi tarihtir.      %
% ---------------------------------------------------------------- %
\tezvermetarih{15 Kasım 2019}{KASIM 2019} 

\tezsavunmatarih{10 Aralık 2019}{ARALIK 2019}

% ---------------------------------------------------------------- %
% Esdanisman ve juri bilgileri asagidadir. Kullanmayacaginiz satiri%
% tumuyle silmeyiniz. Argumanini bos birakiniz.                    %
% Ornek: \esdanismani{}{}                                          %
% ---------------------------------------------------------------- %
\esdanismani{Dr. Kerem PEKER}{Erdemli Proje Müşavirlik}   

\juriBir{Prof. Dr. Elişan Filiz PİROĞLU}{\.Istanbul Teknik \"Universitesi}

\juriIki{Prof. Dr. Bülent AKBAŞ}{Gebze Teknik \"Universitesi}  

\juriUc{}{}

\juriDort{}{}

\juriBes{}{}

\input{defsLyX.tex}

% ---------------------------------------------------------------- %
% Ithaf sayfasinda bulunacak tumceler icin kullanilabilecek        %
% yapi asagidadir. Ithaf sayfasi bulunmayacaksa argumani(ayiraclar %
% arasini) bos birakiniz.                                          %
% ---------------------------------------------------------------- %
\ithaf{Ecem Şengüle,}

% ---------------------------------------------------------------- %
% Asagida belirtilen tex uzantili dosyalarin icerigi               %
% olusturulmalidir.                                                %
% ---------------------------------------------------------------- %

\onsoz{Bu tez çalışması süresince samimiyetini ve hoşgörüsünü benden esirgemeyen,
bilgisi ile her zaman istifade ettiğim danışman hocam Doç. Dr. Barlas
Özden ÇAĞLAYAN’a, mesleki eğitim sürecimin başından beri vizyonu ile
yolumu aydınlatan, üstadım ve eş danışmanım Dr. Müh. Kerem PEKER’e
sonsuz teşekkür ederim.

Çok kıymetli görüşleri ve deneyimlerini benimle paylaşan, İnş. Yük.
Müh. Ahmet KAPTAN’a ve İnş. Yük. Müh. Ahmet Metin YILDIRIM’a teşekkürü
bir borç bilirim.

Bu süreçte desteğini esirgemeyen İnş. Yük. Müh. Ecem ŞENGÜL’e çok
teşekkür ederim.

Tezin hazırlanması aşamasında yardımları için Dr. Ögr. Üyesi Barış
ERKUŞ’a ve mesai arkadaşım İnş. Yük. Müh. Mehmet GEZER’e teşekkür
ederim.

Ayrıca bu yüksek lisans tezi 42127 numaralı İTÜ-Bilimsel Araştırma
Projesi kapsamında kabul görmüştür, destekleri için İstanbul Teknik
Üniversitesi’ne teşekkürlerimi sunarım.

Hayatım boyunca bana her zaman güvenen, koşulsuz destekleyen ve her
daim yanımda hissettiğim aileme, annem Kadriye KARABACAK, babam İbrahim
KARABACAK’a ve kardeşlerime teşekkür ederim.}
\kisaltmalistesi{\hspace{-3mm} %
\begin{tabular}{p{2cm}l}
\textbf{ADRS} & \textbf{:} Acceleration Displacement Response Spectrum\tabularnewline
\textbf{AISC} & \textbf{:} American Institute of Steel Construction\tabularnewline
\textbf{ASCE} & \textbf{:} American Society of Civil Engineers \tabularnewline
\textbf{ATC} & \textbf{:} Applied Technology Council \tabularnewline
\textbf{CUREe} & \textbf{:} California Universities for Research in Earthquake Engineering\tabularnewline
\textbf{FEMA}  & \textbf{:} Federal Emergency Management Agency \tabularnewline
\textbf{NEHRP} & \textbf{:} National Eartquake Hazards Reduction Program \tabularnewline
\textbf{PEER}  & \textbf{:} Pacific Earthquake Engineering Research Center \tabularnewline
\textbf{PGA}  & \textbf{:} Peak Ground Acceleration \tabularnewline
\textbf{PGV}  & \textbf{:} Peak Ground Velocity \tabularnewline
\textbf{SAC} & \textbf{:} SEAOC - ATC - CUREe\tabularnewline
\textbf{SEAOC} & \textbf{:} Structural Engineers Association of California\tabularnewline
\textbf{TBDY}  & \textbf{:} Türkiye Bina Deprem Yönetmeliği \tabularnewline
\textbf{UBC} & \textbf{:} Uniform Building Code\tabularnewline
\textbf{YDKT} & \textbf{:} Yük ve Dayanım Katsayıları ile Tasarım\tabularnewline
\end{tabular}
}
\sembollistesi{\global\long\def\tabcolsep{1pt}%
\global\long\def\arraystretch{1.1}%

\hspace{-3mm}%
\begin{longtable}[c]{>{\raggedright}p{1.5cm}l>{\raggedright}p{13.4cm}}
$a_{0}$  & :  & Rayleigh sönüm matrisi için kütle matrisi çarpanı\tabularnewline
$a_{1}$  & :  & Rayleigh sönüm matrisi için rijitlik matrisi çarpanı\tabularnewline
$a_{1}$  & :  & Birinci moda ait modal ivme\tabularnewline
$C_{\mathrm{d}}$  & :  & Yer değiştirme arttırma katsayısı\tabularnewline
$C_{\mathrm{R}}$  & :  & Spektral yer değiştirme oranı\tabularnewline
$\mathbf{C}$  & :  & Sönüm matrisi\tabularnewline
$d$ & : & Bulon çapı\tabularnewline
$d_{1}$ & : & Birinci moda ait modal yer değiştirme\tabularnewline
$d_{\mathrm{c}}$ & : & Kolon en kesit yüksekliği\tabularnewline
$D$  & :  & Dayanım fazlalığı katsayısı\tabularnewline
$E$  & :  & Elastisite modülü\tabularnewline
$F_{\mathrm{b}}$  & :  & Kiriş başlıkları tarafından aktarılan kuvvet\tabularnewline
$F_{\text{y}}$  & :  & Akma gerilmesi\tabularnewline
$F_{\text{S}}$  & :  & Kısa periyot bölgesi için yerel zemin etki katsayısı\tabularnewline
$F_{\text{1}}$  & :  & 1.0 saniye periyot için yerel zemin etki katsayısı\tabularnewline
$g$ & : & Yer çekimi ivmesi\tabularnewline
$h_{\mathrm{i}}$ & : & i'inci katın yüksekliği\tabularnewline
$I$ & : & Bina önem katsayısı\tabularnewline
$I$ & : & Atalet momenti\tabularnewline
$I_{\mathrm{b}}$ & : & Kiriş atalet momenti\tabularnewline
$L$ & : & Kiriş uzunluğu\tabularnewline
$\mathbf{K}$  & :  & Rijitlik matrisi\tabularnewline
$K_{\mathrm{o}}$  & :  & Alın levhalı ve başlık levhalı birleşim bölgesi rijitliği\tabularnewline
$K_{\mathrm{S}}$  & :  & Sekant rijitliği\tabularnewline
$m_{t1}$ & : & Birinci moda ait modal etkin kütle\tabularnewline
$m^{1}$ & : & Boyutsuz moment parametresi\tabularnewline
$M$ & : & Moment\tabularnewline
$M_{\mathrm{A}}$ & : & A ucundaki moment\tabularnewline
$M_{\mathrm{B}}$ & : & B ucundaki moment\tabularnewline
$M_{\mathrm{j,Rd}}$ & : & Birleşimin tasarım momenti dayanımı\tabularnewline
$M_{\mathrm{n}}$ & : & Nihai eğilme momenti dayanımı\tabularnewline
$M_{\mathrm{p}}$ & : & Plastik eğilme momenti kapasitesi \tabularnewline
$M_{\mathrm{pl,Rd}}$ & : & Tasarım plastik momenti dayanımı \tabularnewline
$M_{\mathrm{s}}$ & : & Servis yükleri altındaki moment\tabularnewline
$M_{\mathrm{w}}$ & : & Moment büyüklüğü\tabularnewline
$\mathbf{M}$  & :  & Kütle matrisi\tabularnewline
$N$  & :  & Eksenel kuvvet\tabularnewline
$R$  & :  & Taşıyıcı sistem davranış katsayısı\tabularnewline
$R_{\mathrm{S}}$  & :  & Dayanım fazlalığı çarpanı\tabularnewline
$R_{\mathrm{y}}$  & :  & Akma dayanımı azaltma katsayısı\tabularnewline
$R_{\mathrm{R}}$  & :  & Bağlılık çarpanı\tabularnewline
$R_{\mathrm{\mu}}$  & :  & Süneklik azaltma katsayısı\tabularnewline
$R_{\mathrm{\xi}}$  & :  & Sönüm çarpanı\tabularnewline
$N_{\mathrm{Ed}}$  & :  & Tasarım eksenel kuvveti\tabularnewline
$S_{\mathrm{a}}$ & : & Spektral ivme\tabularnewline
$S_{\mathrm{ae}}$ & : & Elastik spektral ivme\tabularnewline
$S_{\mathrm{de}}$ & : & Elastik tasarım spektral yer değiştirmesi\tabularnewline
$S_{\mathrm{d1}}$ & : & Doğrusal elastik olmayan spektral yer değiştirme\tabularnewline
$S_{\mathrm{DS}}$ & : & Kısa periyot tasarım spektral ivme katsayısı\tabularnewline
$S_{\mathrm{D1}}$ & : & 1.0 saniye periyot için tasarım spektral ivme katsayısı\tabularnewline
$S_{\mathrm{j}}$ & : & Dönme rijitliği\tabularnewline
$S_{\mathrm{S}}$ & : & Kısa periyot harita spektral ivme katsayısı\tabularnewline
$S_{\mathrm{1}}$ & : & 1.0 saniye periyot için harita spektral ivme katsayısı\tabularnewline
$S_{\mathrm{j,ini}}$ & : & Başlangıç dönme rijitliği\tabularnewline
$t_{\mathrm{p}}$ & : & Panel bölgesi kalınlığı\tabularnewline
$t_{\mathrm{w}}$ & : & Kolon gövde kalınlığı\tabularnewline
$T_{\mathrm{A}}$  & :  & Yatay elastik tasarım ivme spektrumu köşe periyodu\tabularnewline
$T_{\mathrm{B}}$  & :  & Yatay elastik tasarım ivme spektrumu köşe periyodu\tabularnewline
$T_{\mathrm{L}}$  & :  & Yatay elastik tasarım spektrumunda sabit yer değiştirme bölgesine
geçiş periyodu\tabularnewline
$T_{\mathrm{p}}$  & :  & Binanın hakim doğal titreşim periyodu\tabularnewline
$V_{\mathrm{d}}$  & :  & Tasarıma esas alınan taban kesme kuvveti\tabularnewline
$V_{\mathrm{e}}$  & :  & Yapının elastik kalması durumunda sistemde oluşacak en büyük taban
kesme kuvveti\tabularnewline
$\left(V_{\mathrm{S}}\right)_{30}$ & :  & Üst 30 metredeki ortalama kayma dalgası hızı\tabularnewline
$V_{\mathrm{t}}$  & :  & Yapı taban kesme kuvveti \tabularnewline
$V_{\mathrm{t1}}$  & :  & İtme analizi sırasında yapının tepesinde elde edilen birinci moda
ait taban kesme kuvveti\tabularnewline
$V_{\mathrm{y}}$  & :  & Yapının akma durumundaki taban kesme kuvveti\tabularnewline
$z$  & :  & Kuvvet kolu\tabularnewline
$w$  & :  & Düzgün yayılı yük\tabularnewline
$\alpha$  & :  & $C_{\mathrm{d}}$ düzeltme katsayısı\tabularnewline
$\delta_{\mathrm{i,max}}$  & :  & Binanın i'inci katındaki etkin göreli kat ötelemelerinin kat içindeki
en büyük değeri\tabularnewline
$w_{\text{i}}$  & :  & Mander modelinde düşey donatıların eksenleri arasındaki uzaklık\tabularnewline
$w_{\text{i}}$  & :  & Yapının $i$. moduna ait açısal frekansı\tabularnewline
$w_{\text{j}}$  & :  & Yapının $j$. moduna ait açısal frekansı\tabularnewline
$\gamma$  & :  & Panel bölgesi kayma deformasyonu\tabularnewline
$\lambda$  & :  & Göreli kat ötelemelerinin sınıflandırılmasında kullanılan amprik katsayı\tabularnewline
$\Delta_{\mathrm{d}}$  & :  & Tasarıma esas alınan taban kesme kuvvetine karşılık gelen tepe yer
değiştirmesi\tabularnewline
$\Delta_{\mathrm{e}}$  & :  & Yapının elastik kalması durumunda sistemde oluşacak tepe yer değiştirmesi\tabularnewline
$\Delta_{\mathrm{y}}$  & :  & Akma dayanımına karşılık gelen tepe yer değiştirmesi\tabularnewline
$\Delta_{\mathrm{max}}$  & :  & Yapının nihai tepe yer değiştirmesi\tabularnewline
$\mu$  & :  & Yer değiştirme süneklik oranı\tabularnewline
$\theta$  & :  & Dönme\tabularnewline
$\theta_{\mathrm{A}}$  & :  & Kirişin A ucundaki dönme\tabularnewline
$\theta_{\mathrm{B}}$  & :  & Kirişin B ucundaki dönme\tabularnewline
$\theta_{\mathrm{b}}$  & :  & Kirişin dönmesi\tabularnewline
$\theta_{C}$  & :  & Kolonun dönmesi\tabularnewline
$\theta_{cy}$  & :  & Birleşim akma dönmesi\tabularnewline
$\theta_{\mathrm{n}}$  & :  & En büyük moment kapasitesindeki dönme\tabularnewline
$\theta_{\mathrm{s}}$  & :  & Servis yükleri altındaki dönme\tabularnewline
$\theta_{\mathrm{u}}$  & :  & Nihai dönme kapasitesi\tabularnewline
$\theta^{1}$  & :  & Boyutsuz dönme parametresi\tabularnewline
$\xi_{i}$  & :  & Yapının $i.$ açısal frekansına karşılık gelen sönüm oranı \tabularnewline
$\xi_{j}$  & :  & Yapının $j.$ açısal frekansına karşılık gelen sönüm oranı \tabularnewline
$\phi_{n1}$  & :  & Binanın tepesinde birinci moda ait mod şekli genliği\tabularnewline
$\varOmega$  & :  & Dayanım fazlalığı katsayısı\tabularnewline
$\varphi$  & :  & Birleşim dönmesi\tabularnewline
$\Gamma$  & :  & Modal katkı çarpanı\tabularnewline
\end{longtable}

\noindent 
\global\long\def\tabcolsep{6pt}%
\global\long\def\arraystretch{1.0}%
}
\ozet{Moment dayanımlı çelik çerçeveli yapılar emniyet,ekonomi ve estetik
gibi yapı tasarımının önemli performans parametreleri ile öne çıkmaktadır.
Moment dayanımlı çelik çerçeveler yüksek süneklik kapasiteleri ile
deprem bölgelerinde az ve orta katlı ofis türü binalarda sıklıkla
kullanılmaktadır. Moment çerçevelerinin yüksek süneklik özellikleri
yatay kuvvetlere maruz kaldığında sistemin enerji sönümm mekanizmasındaki
basitlikte yatmaktadır. Bu sistemlerde özellikle deprem kuvvetleri
altında yapının dayanımında azalma olmadan kirişlerde çok sayıda akma,
elastik olmayan şekil değiştirme ile sağlanır. Bu tip yapılarda plastik
mafsalların kolonlardan önce kirişlerde oluşmasını sağlamak, yapının
sünekliğini arttırıp, göçmesini geciktirmek için güçlü kolon zayıf
kiriş ilkesi uygulanır. Bu kural gerekenden daha büyük kolon kesitlerinin
seçilmesine ve ekonomik olmayan, aşırı güvenli tasarımlarla sonuçlanmaktadır.
Bu sorunu gidermek için özellikle Amerika’da uygulanan yöntem yapıların
sadece dış çerçevelerini yatay yük taşıyan sistemler olarak tasarlamaktır.
Ancak bu yapıların başlıca dezavantajlarından biri, sistemin yeniden
dağılım imkânının sınırlı olmasıdır. 

1994 Northridge ve 1995 Kobe depremleri sırasında rijit birleşimli
moment dayanımlı çelik çerçeveli yapılarda ve özellikle tamamen kaynaklı
birleşimlerde ağır ve beklenilmeyen hasarlar oluşmuş ve bu tip yapıların
tasarım yöntemlerinin gözden geçirilmesine neden olmuştur. Bu keşiflerin
ardından, SAC Ortak Girişimi olarak bilinen profesyonel derneklerin
ve araştırmacılardan oluşan bir konsorsiyum, bu beklenmeyen davranışın
nedenlerini belirlemek ve daha sağlam moment çerçevesi inşası için
öneriler geliştirmek için kapsamlı bir çalışma yürütmüştür. Çatlakların
ve hasarların, temel birleşim geometrisi, temel malzemelerin kontrolünün
eksikliği, iç yapıdaki düşük tokluğa sahip kaynak dolgu metallerinin
kullanılması, kolon başlıklarında divot kırılması, karşılama levhası
kaynaklı hasarlar, yetersiz kalite kontrol ve diğer faktörlerin bir
sonucu olduğu tespit edildi. Ortaya çıkan araştırma, mevcut özel moment
çerçeveleri için yönetmelik tasarım gerekliliklerinin temelini oluşturmuştur.

Moment çerçeveleri büyük yer değiştirme yapabilme özellikleri hem
avantaj hemde dezavantajı olarak söylenebilir. Bir yapıdaki bütün
kolon-kiriş birleşimlerinin tasarımda tam rijit veya ideal mafsallı
kabulünün ne derece doğru olduğu tartışma konusudur. Gerçekte moment
çerçevesinde kullanılan birleşimlerin güçlü veya zayıf eksende olsun,
davranışı incelendiğinde birleşimlerin rijit veya ideal mafsal davranışı
göstermesi için çok özel detaylandırmaların kullanılması şarttır halbuki
kullanılan birleşimlerin çoğunluğu rijit ve mafsallı iki uç sınır
durumun arasında kalmaktadır. Bu birleşim türleri yarı-rijit birleşim
olarak sınıflandırılmaktadır. 

Geçtiğimiz yarım asırlık süreçte yapısal tasarım için mühendislerin
kullandığı bilgisayar destekli yazılım araçları önemli gelişme kaydetmiştir.
Ancak hala çelik yapıların modellenmesi ve tasarımı aşamasında yapılan
bazı kabuller göze çarpmaktadır. Özellikle deprem bölgesinde tasarlanan
bir çelik yapı için güvenlik ve ekonomi amacı büyük önem arz eder.
Deprem yönetmeliklerinde yer alan yapısal tasarım felsefesi, makul
bir oluşma olasılığı olan bir depremde bile mutlak güvenlik ve hasarsızlığın
sağlanamayacağının altını çizmektedir. Bununla birlikte, bazı yapısal
ve yapısal olmayan hasarlara izin verip elastik olmayan enerjiyi sönümleyerek
yapılarda yüksek bir yaşam güvenliği seviyesini ekonomik olarak sağlamayı
hedefler. Bu tasarım felsefesinin bir sonucu olarak, yönetmeliklerde
öngörülen yatay tasarım kuvveti yapıyı elastik aralıkta tutmak için
gereken yatay kuvvetten daha düşüktür. Yapının elastik zarfta tutulması,
yanal harekete maruz kalan tüm yapısal elemanların, herhangi bir kalıcı
deformasyon ve hasar almadan başlangıç durumuna geri dönmesinin garanti
edilmesi anlamına gelir ki bu durumu korumak, uygulanabilir ve rasyonel
olmaktan uzaktır. 

Moment çerçevelerinin tasarımı göz önüne alındığında tasarımı yöneten
ana parametrenin yer değiştirme olduğu ortaya çıkar. Bu çalışmada
literatürde ve deprem yönetmeliklerinde moment dayanımlı çelik çerçeveli
yapıların tasarımında birleşim dönme rijitliği kavramı ve yönetmeliklerin
tasarımda kullanılacak gerçek yer değiştirme için tepki değişitirme
katsayısı kavramı üzerinde durmaktadır. Yer değiştirme arttırma katsayısı,
yapının elastik durumdaki yer değiştirme değerinden, elastik ötesi
durumdaki yer değiştirme değerini elde etmek için kullanılmaktadır.
Bu amaç doğrultusunda az,orta ve yüksek katlı rijit birleşimli moment
çerçeveli yapılar belirlenen deprem düzeyi için tasarlanmış, beş farklı
dönme rijitliği ile doğrusal elastik olmayan statik ve dinamik analizlerle
incelenmiştir. 15 adet çelik çerçeve statik itme analizleri ve tasarım
depremi seviyesinde ölçeklendirilmiş 18 adet gerçek yer hareketi ile
dinamik analizler yapılmıştır. Bu analiz sonuçları ile her bir çelik
çerçeve için yer değiştirme arttırma katsayısı hesaplanmıştır.

Az, orta ve yüksek katlı çerçevelerde, beş farklı birleşim rijitliği
altında, statik analizler sonucu, ortalama 4.84, dinamik analizler
sonucu, ortalama 3.86 olarak hesaplanan $C_{\mathrm{d}}$ katsayıları
AISC ve TBDY-2018 yönetmeliklerinin sırasıyla kullandığı 5.50 ve 8.00
değerinden daha düşüktür. Sonuçlar çelik çerçeveli bir yapıda birleşim
rijitliğinin yapının davranışını değiştirdiğini, dolayısı ile çelik
çerçeveli yapılar için önemli bir tepki değiştrme katsayısı olan yer
değiştirme arttırma katsayısının değiştirdiği anlaşılmaktadır. Bulunan
yer değiştirme arttırma katsayılarındaki farklılık, moment dayanımlı
bir çelik çerçevenin doğrusal olmayan değerlendirmesi yapılması durumunda
yapıda kullanılan birleşimlerin rijitliklerinin göz ardı edilmemesi
gerektiğini göstermektedir. Ayrıca birleşim rijitliğinin azalmasının
yapıların süneklik oranının azaldığını göstermiştir.}
\summary{Moment resisting steel frames structures stand out with important
performance parameters of structure design such as safety, economy
and aesthetics. Moment resisting steel frames with high ductility
capacities are frequently used in low and medium rise buildings in
earthquake zones. The high ductility characteristics of the moment
frames in the simplicity of the system's energy damping mechanism
when subjected to horizontal forces. In these systems, especially
under earthquake forces, a great number of yields in the beams are
achieved by inelastic deformation without reducing the strength of
the structure. In such structures, the principle of strong column-weak
beam is applied to ensure that plastic hinges are formed in the beams
before the columns, to increase ductility and delay the collapse of
the structure. This rule results in the selection of larger column
sections than required and uneconomical, extremely safe designs. In
order to solve this problem, especially in the USA, the method is
to design only the outer frames of the buildings as horizontal load
bearing systems. However, one of the main disadvantages of these structures
is that the redistribution of the system is limited(low redundancy).

During the 1994 Northridge and 1995 Kobe earthquakes, heavy and unexpected
damages occurred in rigid connected moment resisting steel frames
structures, and especially in fully welded connections, leading to
a review of the design methods of such structures. Following this
discovery, a consortium of professional associations and researchers
known as the SAC joint venture conducted extensive work to identify
the causes of this unexpected behavior and develop recommendations
for building a more robust moment framework. Cracks and damages were
found to be the result of basic joint geometry, lack of control of
basic materials, the use of weld filler metals with low toughness
in the internal structure, divot breakage of column heads, damage
caused by the cover plate, inadequate quality control and other factors.
The resulting research formed the basis for regulatory design requirements
for existing special moment frameworks.

Moment frames can be said to have great displacement properties as
both advantages and disadvantages. It is debatable to what extent
all column-beam joints in a structure are fully rigid or ideally hinged
in the design. In fact, when the behavior of the joints used in the
moment frame is in the strong or weak axis, it is necessary to use
very specific detailing in order to show that the joints exhibit rigid
or ideal hinged behavior, whereas the majority of the joints used
are between two rigid and hinged end boundary states. These connection
types of classified as semi-rigid connections.

Over the past half century, computer aided software tools used by
engineers for structural design have made significant progress. However,
there are still some assumptions made during the modeling and design
of steel structures. Especially for a steel structure designed in
the earthquake zone, the aim of safety and economy is of great importance.
The structural design philosophy in the earthquake regulations underlines
that even in an earthquake with a reasonable probability of occurrence,
absolute safety and damage cannot be ensured. However, it aims to
economically provide a high level of safety in buildings by allowing
some structural and non-structural damage and absorbing inelastic
energy. As a result of this design philosophy, the horizontal design
force prescribed in the regulations is lower than the horizontal force
required to keep the structure within the elastic range. Retention
of the structure in the elastic envelope means that all structural
elements subjected to lateral movement are returned to the initial
state without any permanent deformation and damage, which is to be
maintained, far from practical and rational. 

When the design of moment frames is considered, the main parameter
that governs the design is displacement. This study focuses on the
concept of connection rotation stiffness in the design of steel frame
structures with moment resistance in the earthquake regulations, and
the concept of the reaction coefficient of change for the actual displacement
of regulations to be used in design. The \emph{Displacement Amplification
Facto}r is used to obtain the displacement value of the structure
from the elastic displacement value in the inelastic state. For this
purpose, low, medium and high storied rigid connections moment framed
structures were designed for the earthquake level determined and examined
with five different rotational stiffnesses by linear inelastic static
and dynamic analyses. 15 steel frame static pushover analyses and
18 real-ground motion dynamic time-history analyses were scaled at
the design earthquake level. With these analysis results, the displacement
amplification factor was calculated for each steel frame.

In low, medium and high-rise frames, $C_{\mathrm{d}}$ coefficients
calculated under average of 4.84 for dynamic analysis and 3.86 for
dynamic analysis are less than the values of 5.50 and 8.00 used by
AISC and TBDY-2018 regulations, respectively. The results show that
the stiffness of the connection in a steel-frame structure changes
the behavior of the structure, thereby changing the \emph{Displacement
Amplification Factor}, which is an important response coefficient
for steel-frame structures. The difference in the displacement amplification
coefficients shows that the rigidity of the connections used in the
structure should not be ignored if a non-linear evaluation of a moment
resisting steel frame is performed. In addition, the decrease in connection
stiffness has shown that the ductility ratio of the structures is
reduced.
}

%------------------------------------%
% Packages added by BE
\usepackage{hypenTR}
\usepackage{longtable}
%------------------------------------%

%------------------------------------------------------------------%
% Matrisler için bazı ek araçlar                                   %
% Eğer \input olarak girilmiş LyX dosyalarında kullanılacaksa,     %
% bunları o dosyaya kopyalayarak "Instant Preview ile denklemler   %
% görülebilir                                                      %
%                                                                  %
\usepackage{upgreek}
\usepackage{pdfrender}
\newcommand*{\boldgreek}[1]{%
  \textpdfrender{%
    TextRenderingMode=FillStroke,%
    LineWidth=.35pt,%
  }{#1}%
}
%
% BE, 2018-10-05
%------------------------------------------------------------------%

% ---------------------------------------------------------------- %
% Custom commands
% ---------------------------------------------------------------- %
%\renewcommand*\arraystretch{1.5}   % To increase the row height of tables
%Heceleme için hyphenpenalty değerini küçültebilirsiniz. 
%Değer küçüldükçe hecelemek isteyecek, değer büyükdükçe hecelemekten kaçacaktır.
%This block moved to the LaTeX file for convenience
\hyphenpenalty=10000
\exhyphenpenalty=10000
\widowpenalty=10000
\clubpenalty=10000

\makeatother

\begin{document}

\chapter{GİRİŞ}

\label{CH1}

Deprem bölgelerinde yapılacak binalarda yaygın olarak kullanılan kuvvet
esaslı tasarıma göre, yapının karşılaşması beklenen en büyük deprem
kuvveti altında göreli olarak büyük yer değiştirmeler yapması, yapısal
taşıyıcı elemanlarda hasar oluşması ve dolayısıyla deprem enerjisinin
büyük bir kısmının kalıcı şekil değiştirmeler ile sönümlenmesi beklenir.
Yapının önem derecesine bağlı olarak depremde daha az hasar almasını
sağlamak amacıyla göreli kat ötelemelerinin sınırlanması gerekmektedir.
Bu durum, yapının rijitliğini artırarak katlarda daha büyük ivmelerin
oluşmasına ve buna bağlı olarak yapısal olmayan elemanlarda daha büyük
kuvvetlerin oluşmasına sebep olmaktadır. Kat ivmelerinin azaltılması
ise yapı rijitliğinin düşürülmesi ile mümkündür. Ancak bu durumda
göreli kat ötelemeleri ve yapısal hasarlar artmaktadır. Yapıların
depremden korunmasının bir yolu, zemin ile yapı temeli arasında, yatay
rijitliği yapıya oranla düşük bir katman oluşturmaktır.

Yapılarda kuvvetli yer hareketleri sebebiyle oluşan hasarları engellemek
amacıyla birçok yöntem üzerine çalışmalar yapılmıştır. Touaillon 1870
yılında, ABD patent ofisine yaptığı başvuruda, temel ve binaya sabitlenen
konkav metal plakaların arasında bulunan küreler sayesinde deprem
sebebiyle yapıların yıkılmasının önleneceğini söylemektedir \cite{Touaillon1870}.
Benzer bir yalıtım sistemi ise Bechtold tarafından 1907 yılında patent
başvurusu yapılan, binaların altına yerleştirilen rijit taban plakasının
bazalt, gnays veya granitten yapılacak küreler üzerinde kayan bir
sistem önerisidir \cite{Bechtold1907}. Dr. Calantarines ise yapı
temelinin ince kum, mika veya talk üzerinde kayabildiği bir ``serbest
mesnet'' fikri ile 1909 yılında İngiltere patent ofisine başvurmuştur
\cite{calantarients1909improvements}.

Yeni yöntemlerin araştırılmasının yanında üst katlarda önemli hasarların
oluşmasını engellemek amacıyla yumuşak ilk kata sahip yapılar da depreme
etkileri altında incelenmiştir. Chopra ve diğerleri tarafından yapılan
çalışmada ilk katı yumuşak kata sahip sekiz katlı bir binanın, deprem
etkileri altında üst katlarında hasar oluşmasını engellemek amacıyla
ilk katın akma koşulları incelenmiştir \cite{doi:10.1002/eqe.4290010405}.

Kauçuk mesnetler, binalarda titreşimleri engellemek ve köprülerde
ise ısıl genleşmeler sebebiyle mesnetlerin yer değiştirebilmelerini
sağlamak amacıyla kullanılmıştır. Yapıların deprem etkilerinden korunmaları
amacıyla ilk kullanımı 1969 yılında Yugoslavya'nın Skopje ilinde bulunan
Pestalozzi ilkokulunda gerçekleşmiştir. Tek bir blok olarak uygulanan
kauçuk mesnetlerin yatay ve düşey rijitliklerinin aynı olması, yan
yüzeylerde binanın ağırlığından dolayı şişkinlik oluşmasına neden
olmuştur. Kauçuk katmanların eksenel taşıma kapasitelerinin yükseklikleri
ile ters orantılı olduğunu tespit eden Fransız mühendis Eugène Freyssinet,
kauçuk katmanların aralarına eksenel kuvvete dik doğrultuda ince çelik
plakalar ilave ederek güçlendirmeyi önermiştir. Burada katmanlar arasındaki
bağ sürtünme kuvveti sebebiyle sağlanmaktadır. İnce çelik plakalar
ile kauçuk katmanların birbirlerine yapışmalarını sağlamak amacıyla
kullanılan vulkanizasyon yöntemi sayesinde modern haline kavuşan sismik
izolatörler ile ilgili yapılan çalışmalar ve uygulama örnekleri artmaya
başlamıştır. Kelly tarafından yapılan çalışmada 1900-1984 yılları
arasında sismik yalıtım çalışmaları özetlenerek alfabetik ve kronolojik
bibliyografya sunulmuştur \cite{KELLY1986202}.

Robinson ve Tucker tarafından gerçekleştirilen çalışmada, Şekil \ref{fig:LRBIsolator}'de
gösterilen çelik plakalar ile güçlendirilmiş kauçuk izolatörün merkezine
kurşun silindir yerleştirilerek tekrarlı yatay yükler etkisindeki
davranışı incelenmiştir \cite{RobinsonTucker1977}. Çelik ve kauçuk
ile sargılı olan kurşun silindirde tamamen kayma deformasyonları meydana
gelmektedir. Çalışmada, günümüzde yaygın olarak kullanılan bu sistemin
sahip olduğu histeretik sönüm kapasitesi, kurşun çekirdeğin eksenel
dayanımı ve geri çağırım kuvveti gibi faydaları belirtilerek büyük
deprem etkileri ve küçük rüzgar yüklerinde yeterli performans sergileyeceği
vurgulanmıştır.
\begin{figure}[h!]
\centering{}\includegraphics[width=6.5cm,height=6cm]{TikZ/Grafik_1}\hspace{1cm}
\includegraphics[width=6.5cm,height=6cm]{TikZ/Grafik_1}\caption{\label{fig:LRBIsolator} Kurşun çekirdekli kauçuk izolatör.}
\end{figure}

Yapıların mesnetleri üzerinde sarkaç gibi hareket etmelerini sağlayan
sismik yalıtım yöntemi Zayas ve diğerleri tarafından 1990 yılında
yapılan çalışmalar neticesinde önerilmiştir \cite{doi:10.1193/1.1585573}.
Şekil \ref{fig:FPIsolator}'de kesiti verilen sürtünmeli sarkaç tipi
izolatörlerin yalıtım periyotları konkav yüzeyin yarıçapı ile belirlenmektedir.
Deprem enerjisinin sönümlendiği histeretik davranış, yüzeylerde oluşan
sürtünme kuvvetleri nedeniyle oluşmaktadır. Sistemin sarkaç hareketine
başlaması, sürtünme kuvvetlerinin aşılmasıyla mümkün olmaktadır. Bu
nedenle yapının rüzgar yükleri altında yalıtım birimlerinden beklenen
rijitlik elde edilmiş olmaktadır. Sürtünmeli sarkaç tipi izolatörlerin
malzeme belirsizliklerinden en az düzeyde etkilenmesi, sismik yalıtımlı
yapıların deprem etkileri altındaki davranışlarının öngörülebilir
olmasını sağlamaktadır. Ayrıca yapılan testlere göre büyük deplasmanlarda
eksenel taşıma kapasitesinde düşme veya stabilite kaybı olmadığı ve
histeretik çevrimler ile dayanım azalmasının oluşmadığı görülmüştür.
\begin{figure}[h!]
\centering{}\includegraphics{TikZ/FPIsolator} \caption{\label{fig:FPIsolator}Sürtünmeli sarkaç tipi izolatör.}
 
\end{figure}

Sismik izolasyon, dünyada depremselliği yüksek olan bölgelerde yoğun
olarak kullanılmaktadır. Japonya, Çin, Rusya, İtalya ve ABD başta
olmak üzere 30 dan fazla ülkede 23.000'i aşkın bina, sismik izolasyon
ve sönümleyiciler ile korunmaktadır \cite{Martelli2014}. Aktif fay
hatları üzerinde bulunan ülkemizde ise sismik izolasyon kullanımı
giderek yaygınlaşmaktadır. Ağırlıklı olarak depremden hemen sonra
kesintisiz kullanımın hedeflendiği hastaneler ve veri merkezleri gibi
önemli binalarda uygulanmaktadır. İstanbul'da bulunan Sabiha Gökçen
Uluslararası Havalimanında üç sürtünme yüzeyli sarkaç tipi izolatörler
kullanılarak sismik yalıtım uygulanmıştır. Toplamda 160.000 $\mathrm{m^{2}}$'den
fazla alana kurulu havalimanında 252 sismik izolatör bulunmaktadır.
Üç sürtünme yüzeyli sismik izolatörler ve çelik üst yapının montajı
2008'de tamamlanmıştır (Şekil \ref{fig:SabihaG=0000F6k=0000E7en}).
\begin{figure}[h!]
\centering{}\includegraphics[height=5cm]{fig/a.PNG} \hspace{1cm}
\includegraphics[height=5cm]{fig/b.PNG} \vspace{6pt}
 \caption{\label{fig:SabihaG=0000F6k=0000E7en} Sabiha Gökçen Uluslararası Havalimanına
ait sismik izolatör kolon kiriş birleşimi ve çelik üst yapı \cite{Zekioglu2009}.}
\end{figure}

\newpage{}

\section{Problemin Tanımı}

Sismik yalıtımlı yapıların tasarımı, doğrusal yöntemlerle ön boyutlandırma
aşaması ve doğrusal olmayan dinamik analizlerle tasarımın doğrulanması
olmak üzere iki aşamalıdır. Ön tasarım aşamasında eşdeğer doğrusal
analiz ve mod birleştirme yöntemleri kullanılmaktadır.

Sismik yalıtımlı yapılarda izolatörlerin hakim frekansları ile tabanı
ankastre kabul edilen üst yapı frekansları belirgin biçimde ayrıklaşır.
Göreli olarak düşük yatay rijitliğe sahip yalıtım birimleri, üst yapının
yapacağı deformasyonları azaltmaktadır. Bu nedenlerle elastik sınırlar
içerisinde davranması beklenen yapının, yapısal düzensizliklerinin
bulunmaması halinde gerekli idealleştirmeler yapılarak sistem basitleştirilebilir.
Ayrıca yapının sahip olduğu doğrusal olmayan dinamik özellikler, idealleştirilmiş
sisteme karşılık gelen parametrelerle ifade edilebilir. Eşdeğer doğrusal
analiz yönteminde tüm yapı, Şekil \ref{fig:equivalentmodel}'de belirtildiği
gibi tek serbestlik dereceli bir sistem olarak temsil edilmektedir.
\begin{figure}[h!]
\centering{}\includegraphics{TikZ/StructuralEquivalentModel} \caption{\label{fig:equivalentmodel} Eşdeğer Tek Serbestlik Dereceli Sistem.}
\end{figure}

Sismik yalıtımlı yapı sisteminin, tasarlanacağı yönetmelikte eşdeğer
doğrusal analiz için belirtilen uygulama sınırları dışında kalması
durumunda veya üst yapı taşıyıcı sisteminin sahip olduğu serbestlik
dereceleri de dikkate alınarak daha detaylı bir çözüm yapılmak istendiğinde
mod birleştirme yöntemi tercih edilmektedir. Üst yapının çok serbestlik
dereceli sistem olarak modellendiği bu yöntemde, izolatörlerin temsil
edildiği doğrusal kesme yaylarında eşdeğer rijitlik değerleri kullanılmaktadır.
Böylece kütlesi ve rijitliği belli olan sistemin doğrusal mod şekilleri
ve bu modlara karşılık gelen periyot ve frekans değerleri hesaplanır.
Tüm modlardan elde edilen iç kuvvet ve yer değiştirmeler birleştirilerek
sistemin deprem etkileri hesaplanır.

Doğrusal yöntemlerle tasarımı tamamlanan sismik yalıtımlı yapıların
kontrolü, geçmiş depremlerden elde edilen veya bölgenin depremsellik
özelliklerine uygun biçimde yapay olarak üretilen yer ivmeleri etkisinde,
yapıların doğrusal olmayan davranışlarının modellendiği, zaman tanım
alanında yapılan analizler ile sağlanır. İzolatör birimleri için kullanılan
kesme yaylarında doğrusal olmayan davranışların çeşitli malzeme modelleri
ile ifade edilebileceği gibi, eksenel yaylar, eğilme ve burulma yayları
da tanımlanabilmektedir. Ayrıca bu yayların birbirleri ile bağlı olarak
çalışması gerçeğe daha yakın analiz sonuçlarının elde edilmesini sağlamaktadır.
Bu analizde yapısal davranışlar, kütle ve rijitliklerin yanında yer
hareketlerinin içeriğine de bağlı olarak değişkenlik göstermektedir.
Bu nedenle yapının doğrusal olmayan dinamik analizinde, tasarımda
uyulan yönetmelik koşullarında belirtilen sayıda ve tasarım spektrumuna
göre ölçeklenmiş veya eşlenmiş yer ivmesi kullanılarak bulunan sonuçların
ortalaması değerlendirilmektedir.

Sismik yalıtımlı yapıların tasarım metotları 1970'lerden günümüze
kadar fiziksel deneyler veya analitik modellerden elde edilen bilgiler
ışığında pek çok kez irdelenmiştir. Yapılan araştırmalar içinde doğrusal
tasarım metotlarının geçerliliği, kullanım sınırları ve doğruluk mertebeleri
incelenen konular arasında bulunmaktadır. Sismik yalıtımlı yapıların
tasarımında kullanılan doğrusal yöntemler, konvansiyonel yapılardan
farklı olarak yalnızca ön tasarım aşamasında kullanılmaktadır. Öngörülen
en büyük depremden sonra dahi yapının kesintisiz olarak kullanılma
gerekliliği, yapısal tasarımda gerçeğe en yakın sonuçların elde edilmesini
sağlayacak analizlerin uygulanma zorunluluğunu da beraberinde getirmektedir.
Dolayısıyla ana hatları doğrusal analizler ile belirlenen sismik yalıtımlı
sistem tasarımlarının, bölgenin depremselliğine uygun olarak seçilecek
deprem kayıtları kullanılarak yapılacak doğrusal olmayan dinamik analizler
sayesinde kontrol edilerek son hale getirilmesi önemlidir. Fakat bu
ileri seviye analizler, karmaşık olabilmekte ve uzun zaman almaktadır.
Bu nedenle, son aşamaya kadar doğrusal yöntemler ile ilerletilen sismik
yalıtımlı yapı projelerinde bu analizlerin doğruluğu önem arz etmektedir.
İlerleyen başlıklarda doğrusal analiz yöntemleri irdelenmiştir.

\section{Amaç ve Kapsam}

\label{EquivalentLinearAnalysis} Eşdeğer doğrusal analiz yönteminde
kullanılan sistemde izolatörlerin üzerinde bulunan yalıtım düzlemi
ve üst yapı, tek bir kütle olarak değerlendirilmektedir. İzolatör
eşdeğer rijitliği, her bir izolatör biriminin tasarım deplasmanı için
elde edilecek sekant rijitlikleri toplamını belirtmektedir. Buna bağlı
olarak izolatör periyodu, tek serbestlik dereceli sistem için hesaplanmaktadır.
Sistemin sönümleyeceği enerji ise izolatörlerin doğrusal olmayan histeretik
davranışlarından elde edilecek eşdeğer viskoz sönüm oranı cinsinden
ifade edilir. Eşdeğer doğrusal analiz yöntemi, içerdiği kabul ve idealleştirmeler
sebebi ile kullanılan yönetmeliklerde belirtilen kriterlerin sağlanması
durumunda uygulanabilmektedir.

Ön tasarım aşamasında izolatörlerde meydana gelecek en büyük yatay
yer değiştirme değerinin tespit edilmesi izolatör boyutlarının belirlenmesinde
önemli bir kriterdir. En büyük yer değiştirme değeri, en büyük izolatör
kuvvetinin eşdeğer rijitliğe bölümü ile bulunmaktadır. Fakat eşdeğer
rijitlik değerinin belirlenebilmesi, izolatörün en büyük yer değiştirme
değerinin bilinmesi ile mümkün olmaktadır. Ön tasarımın ilk adımında
oluşan bu belirsizlik, detayları Bölüm \ref{CH2}'de sunulan iterasyonlar
ile ortadan kaldırılmaktadır.

Üst yapının ön tasarımı için gerekli taban kesme kuvveti, bu yöntemde
yalıtım düzleminde oluşan kesme kuvvetine göre hesaplanmaktadır. Tasarımda
takip edilen yönetmeliğe bağlı olarak taban kesme kuvvetinin belirlenmesi
ve bu kuvvetin üst yapı katlarına dağıtılma biçimi farklılık göstermektedir.
EN 1998-1:2004'e göre, katlara etkiyen yatay kuvvetler eşdeğer periyot
ve sönüme göre hesaplanacak spektral ivmenin, ilgili kat kütlesi ile
çarpımından elde edilmektedir\cite{Eurocode2004}. Burada kat kütlelerinin
eşit olduğu kabulü yapılırsa kesme kuvvetlerinin katlara eşit miktarda
dağıtıldığı görülmektedir. Ayrıca yalıtım birimi seviyesinde meydana
gelen kesme kuvvetinin yapı kütlesine oranı ile üst yapı taban kesme
kuvvetinin üst yapı toplam kütlesine oranı eşit olmaktadır.

ASCE/SEI-7-10 (2010) ve TBDY (2018) yönetmeliklerinde yer alan formüle
göre taban kesme kuvveti, yalıtım seviyesinde oluşan kesme kuvvetine
eşit alınarak üst yapı katlarına üçgen formunda dağıtılmaktadır \cite{ASCE2010,TBDY2018}.

ASCE/SEI-41-13 (2014) ve ASCE/SEI-7-16 (2016) yönetmeliklerinde ise
yalıtım düzleminde oluşacak kesme kuvveti, yapı taban kesme kuvvetine
eşit olarak alınmamaktadır \cite{ASCE41-13,ASCE2016}. York ve Ryan
tarafından yapılan çalışma, eşdeğer viskoz sönüm oranının artması
halinde, yapı taban kesme kuvvetinin yalıtım düzleminde oluşan kesme
kuvvetine oranının da artacağını göstermektedir \cite{doi:10.1080/13632460802003751}.
Buna bağlı olarak önerilen formül, istatistiksel parametreyi temsil
eden katsayıda bulunan değişiklik ile birlikte bu güncel yönetmeliklerde
yer almaktadır. Elde edilen artırılmış taban kesme kuvveti ise katlara
çalışmada önerilen formülün güncel haline göre paylaştırılmaktadır.

TBDY (2018) yönetmeliğine göre hesaplanacak taban kesme kuvvetinde
belirlenecek deprem yükü azaltma katsayısı $R$, hedeflenecek kesintisiz
kullanım ve sınırlı hasar performans düzeyleri için dayanım fazlalığı
katsayısı $D$ ile eşit verilmektedir \cite{TBDY2018}. Benzer olarak
ASCE/SEI-7-10 (2010) yönetmeliğinde belirtilen formülde yer alan $R_{1}$
katsayısı, üst yapı yatay taşıyıcı sistemine bağlı olarak belirlenecek
$R$ katsayısının $3/8$'i alınarak hesaplanmakta ve en fazla $2$
olarak belirlenebilmektedir \cite{ASCE2010}. Dolayısıyla $R_{1}$
katsayısı, hiçbir taşıyıcı sistem için dayanım fazlalığı katsayısı
$\Omega_{\text{0}}$'dan büyük olmamaktadır. Böylece yapının karşılaşması
beklenen en büyük deprem etkisi altında elastik sınırlar içinde kalması,
yalnızca eşdeğer doğrusal analiz yönteminin belirlediği kabuller çerçevesinde,
yalıtım düzlemi kesme kuvvetinin üst yapı taban kesme kuvvetine eşit
olması durumunda mümkün olmaktadır.



\chapter{YARI-RİJİT BİRLEŞİMLER}

\label{CH2} 

\section{Birleşimlerin Sınıflandırılması}

Bu bölümde, ortak yalıtım düzleminde bulunan sismik yalıtımlı tekli
ve çoklu yapılara ait hareket denklemleri, çalışmada kullanılacak
hali ile yeniden türetilmiş ve ardından hareket denklemlerinin çalışma
kapsamında kullanılan doğrusal olmayan çözüm yöntemine ait detaylar
verilmiştir. Bununla birlikte sismik yalıtımlı yapıların analizinde
kullanılan sönüm modellerine ait detaylar paylaşılmıştır. Ayrıca sismik
yalıtımlı yapıların eşdeğer doğrusal analizine ilişkin tasarım metodolojileri
ilgili yöntem ve yönetmelikler dahilinde açıklanmıştır.

\section{Birleşim Davranışının Modellenmesi}

\label{eom-singlestructure} Nagarajaiah ve diğerleri tarafından sunulan
sismik yalıtımlı tek yapılara ait hareket denklemlerinde, her bir
kat 3 serbestlik derecesi ile ifade edilmektedir \cite{doi:10.1061/(ASCE)0733-9445(1991)117:7(2035)}.
Bu çalışma kapsamında, tek doğrultuda yatay ötelenme serbestlik dereceleri
dikkate alınacaktır. Bu sebeple hareket denklemleri, çalışma kapsamında
kullanılacak sınırlar çerçevesinde detaylı olarak açıklanmıştır. Denklemler,
kütleleri kat hizalarında toplanmış olan ve yalnızca yatay kesme yaylarının
yapı rijitliğini temsil ettiği örnek sismik yalıtımlı sistem üzerinden
türetilmiştir. 
\begin{figure}[h!]
\centering{}\includegraphics{TikZ/SingleStructureAndFBD} \caption{\label{fig:singlestructurefbd}Sismik yalıtımlı tek yapının şekil
değiştirmiş hali ve serbest cisim diyagramı.}
\end{figure}

Şekil \ref{fig:singlestructurefbd}'de yapının temsili şekil değiştirmiş
hali ile dinamik yükleme altında katlara ve yalıtım düzlemine etkiyen
kuvvetler sunulmuştur. Burada, $x^{\text{b}}$ katların yalıtım düzlemine
göre bağıl yer değiştirmelerini, $x_{\text{b}}^{\text{g}}$ yalıtım
düzleminin zemine göre bağıl yer değiştirmesini, $m$ kat kütlelerini,
$m_{\text{b}}$ yalıtım düzlemi kütlesini, $k$ kat rijitliklerini,
$c$ ilgili kata ait sönüm değerini, $F_{\text{s}}$ yapının taban
kesme kuvvetini, $F_{\text{iso}}$ izolatör seviyesinde oluşan kesme
kuvvetini ve noktalar ise zamana bağlı türevi belirtmektedir. Terimlerde
alt indis şeklinde bulunan numaralar, ilgili büyüklüğün hangi kata
ait olduğunu göstermektedir. Buna göre, yapının ilk katına ait hareket
denklemi \ref{eq-eom-1}'deki gibi ifade edilebilir. 
\begin{equation}
-k_{2}\left[x_{2}^{\text{b}}-x_{1}^{\text{b}}\right]+k_{1}x_{1}^{\text{b}}-c_{2}\left[\dot{x}_{2}^{\text{b}}-\dot{x}_{1}^{\text{b}}\right]+c_{1}\dot{x}_{1}^{\text{b}}=-m_{1}\ddot{x}_{1}^{\text{abs}}\label{eq-eom-1}
\end{equation}
Dinamik denge, D'Alembert prensibine göre yazılmıştır. Bu prensibe
göre, sisteme etkiyen kuvvetlerin yanı sıra, fiktif atalet kuvvetlerinin
ivme yönüne ters olarak eklenmesi halinde sistem her zaman anı için
denge durumunda bulunmaktadır. Dolayısıyla eşitliğin sağ tarafında
yer alan $\ddot{x}_{1}^{\text{abs}}$ terimi, aşağıdaki denklemde
belirtildiği gibi yalıtım düzleminde oluşan ve $\ddot{x}_{\text{b}}^{\text{abs}}$
ile ifade edilen ivmeler ile, dinamik dengenin sağlanabilmesi için
denkleme eklenen fiktif $\ddot{x}_{1}^{\text{b}}$ ivmesinin toplamını
ifade etmektedir. 
\begin{equation}
\ddot{x}_{1}^{\text{abs}}=\ddot{x}_{1}^{\text{b}}+\ddot{x}_{\text{b}}^{\text{abs}}\label{eq-acc-1}
\end{equation}
Benzer şekilde yapının ikinci katına ait hareket denklemi ve $\ddot{x}_{2}^{\text{abs}}$
teriminin açılımı aşağıda sunulmuştur. 
\begin{equation}
k_{2}\left[x_{2}^{\text{b}}-x_{1}^{\text{b}}\right]+c_{2}\left[\dot{x}_{2}^{\text{b}}-\dot{x}_{1}^{\text{b}}\right]=-m_{2}\ddot{x}_{2}^{\text{abs}}\label{eq-eom-2}
\end{equation}
\begin{equation}
\ddot{x}_{2}^{\text{abs}}=\ddot{x}_{2}^{\text{b}}+\ddot{x}_{\text{b}}^{\text{abs}}\label{eq-acc-2}
\end{equation}
Yukarıda verilen denklemler düzenlenerek matris formunda yazılırsa
aşağıda sunulan denklem elde edilir. 
\begin{equation}
\begin{split}\begin{bmatrix}m_{1} & 0\\
0 & m_{2}
\end{bmatrix}\begin{Bmatrix}\ddot{x}_{1}^{\text{b}}\\
\ddot{x}_{2}^{\text{b}}
\end{Bmatrix}+ & \begin{bmatrix}c_{1}+c_{2} & -c_{2}\\
-c_{2} & c_{2}
\end{bmatrix}\begin{Bmatrix}\dot{x}_{1}^{\text{b}}\\
\dot{x}_{2}^{\text{b}}
\end{Bmatrix}+\begin{bmatrix}k_{1}+k_{2} & -k_{2}\\
-k_{2} & k_{2}
\end{bmatrix}\begin{Bmatrix}{x}_{1}^{\text{b}}\\
{x}_{2}^{\text{b}}
\end{Bmatrix}=\\
 & -\begin{bmatrix}m_{1} & 0\\
0 & m_{2}
\end{bmatrix}\begin{Bmatrix}1\\
1
\end{Bmatrix}\ddot{x}_{\text{b}}^{\text{abs}}
\end{split}
\label{eq-eom-singlestructure}
\end{equation}
Aşağıdaki denklemde açıklandığı üzere $\ddot{x}_{\text{b}}^{\text{abs}}$
terimi, $\ddot{x}_{\text{b}}^{\text{g}}$ ile ifade edilen yalıtım
düzleminin zemine göre bağıl ivmesi ile $\ddot{x}_{\text{g}}^{\text{abs}}$
ifadesine karşılık gelen yer ivmesinin toplamını belirtmektedir. 
\begin{equation}
\ddot{x}_{\text{b}}^{\text{abs}}=\ddot{x}_{\text{b}}^{\text{g}}+\ddot{x}_{\text{g}}^{\text{abs}}\label{eq-abs-base-acc}
\end{equation}
Yalıtım düzlemine göre bağıl olarak ifade edilmiş olan yapının hareket
denklemi kapalı formda aşağıdaki gibi yazılabilir. 
\begin{equation}
\mathbf{M}_{\text{s}}\mathbf{\ddot{x}}_{\text{s}}^{\text{b}}+\mathbf{C}_{\text{s}}\mathbf{\dot{x}}_{\text{s}}^{\text{b}}+\mathbf{K}_{\text{s}}^{\text{b}}\mathbf{x}_{\text{s}}^{\text{b}}=-\mathbf{M}_{\text{s}}\mathbf{R}\ddot{x}_{\text{b}}^{\text{g}}-\mathbf{M}_{\text{s}}\mathbf{R}\ddot{x}_{\text{g}}^{\text{abs}}\label{eq-eom-singlestructure-closed}
\end{equation}
Burada, $\mathbf{M}_{\text{s}}$ kütle, $\mathbf{C}_{\text{s}}$ sönüm,
$\mathbf{K}_{\text{s}}$ rijitlik matrisini, $\mathbf{x}_{\text{s}}$
katların yalıtım düzlemine göre bağıl yer değiştirme vektörünü, $\mathbf{\ddot{x}}_{\text{b}}^{\text{g}}$
yalıtım düzleminin zemine göre bağıl ivmesini, $\mathbf{\ddot{x}}_{\text{g}}^{\text{abs}}$
yer ivmesini, $\mathbf{R}$ deprem etki vektörünü ve noktalar ise
zamana bağlı türevleri ifade etmektedir. Elde edilen üst yapı bağıl
hareket denkleminde deprem kuvvetlerini oluşturacak ivmeler, yalnızca
yalıtım düzleminin mutlak ivmesine eşittir. Dolayısıyla üst yapıya
etkiyen deprem kuvvetlerinde yer ivmelerinin doğrudan etkisi bulunmamaktadır.

Burada, $\mathbf{M}$, $\mathbf{C}$ ve $\mathbf{K}$ tüm yapıya ait
sırasıyla kütle, sönüm ve rijitlik matrislerini, $\mathbf{x}$ yapı
yer değiştirme vektörünü ve $\mathbf{S}_{\text{1}}$ ise etki vektörünü
ifade etmektedir. Denklemde belirtilen terimlerin açık hali aşağıda
verilmiştir. \vspace{0.2cm}

\begin{center}
$\mathbf{M}=\begin{bmatrix}\mathbf{M}_{\text{s}} & \mathbf{M}_{\text{s}}\mathbf{R}\\
\mathbf{R}^{\text{T}}\mathbf{M}_{\text{s}} & \mathbf{R}^{\text{T}}\mathbf{M}_{\text{s}}\mathbf{R}+M_{\text{b}}
\end{bmatrix}$ \hspace{0.2cm} $\mathbf{M}_{\text{s}}=\begin{bmatrix}m_{1} & 0\\
0 & m_{2}
\end{bmatrix}$ \vspace{0.25cm}
\par\end{center}

\begin{center}
$\mathbf{C}=\begin{bmatrix}\mathbf{C}_{\text{s}} & \mathbf{0}\\
\mathbf{0} & C_{\text{b}}
\end{bmatrix}$ \hspace{0.2cm} $\mathbf{C}_{\text{s}}=\begin{bmatrix}c_{1}+c_{2} & -c_{2}\\
-c_{2} & c_{2}
\end{bmatrix}$ \vspace{0.25cm}
\par\end{center}

\begin{center}
$\mathbf{K}=\begin{bmatrix}\mathbf{K}_{\text{s}} & \mathbf{0}\\
\mathbf{0} & K_{\text{b}}
\end{bmatrix}$ \hspace{0.2cm} $\mathbf{K}_{\text{s}}=\begin{bmatrix}k_{1}+k_{2} & -k_{2}\\
-k_{2} & k_{2}
\end{bmatrix}$ \vspace{0.25cm}
\par\end{center}

\begin{center}
$\mathbf{\ddot{x}}=\begin{Bmatrix}\mathbf{\ddot{x}}_{\text{s}}^{\text{b}}\\
\ddot{x}_{\text{b}}^{\text{g}}
\end{Bmatrix}$ \hspace{0.2cm} $\mathbf{\ddot{x}}_{\text{s}}^{\text{b}}=\begin{Bmatrix}\ddot{x}_{1}^{\text{b}}\\
\ddot{x}_{2}^{\text{b}}
\end{Bmatrix}$ \hspace{0.5cm} $\mathbf{\dot{x}}=\begin{Bmatrix}\mathbf{\dot{x}}_{\text{s}}^{\text{b}}\\
\dot{x}_{\text{b}}^{\text{g}}
\end{Bmatrix}$ \hspace{0.2cm} $\mathbf{\dot{x}}_{\text{s}}^{\text{b}}=\begin{Bmatrix}\dot{x}_{1}^{\text{b}}\\
\dot{x}_{2}^{\text{b}}
\end{Bmatrix}$ \hspace{0.5cm} $\mathbf{x}=\begin{Bmatrix}\mathbf{x}_{\text{s}}^{\text{b}}\\
{x}_{\text{b}}^{\text{g}}
\end{Bmatrix}$ \hspace{0.2cm} $\mathbf{x}_{\text{s}}^{\text{b}}=\begin{Bmatrix}{x}_{1}^{\text{b}}\\
{x}_{2}^{\text{b}}
\end{Bmatrix}$ \vspace{0.25cm}
\par\end{center}

\begin{center}
$\mathbf{S}_{\text{1}}=\begin{Bmatrix}\mathbf{0}\\
1
\end{Bmatrix}$ \hspace{0.2cm} $\mathbf{R}=\begin{Bmatrix}1\\
1
\end{Bmatrix}$ 
\par\end{center}

\newpage{}

\section{Birleşim Dönme Kapasitesi}

Sismik yalıtımlı çoklu yapılara ait hareket denklemleri Tsopelas ve
diğerleri tarafından açıklanmıştır \cite{TSOPELAS199447}. Yapılan
çalışmada, düzlem içi rijitlikleri sonsuz yapı katlarını temsil eden
toplu kütlelerin iki yatay ötelenme ve bir dönme olmak üzere toplam
üç adet serbestlik derecesi bulunmaktadır. Kütle merkezlerinde konumlanan
bu serbestlik dereceleri, yalıtım düzlemi kütle merkezinden geçen
düşey referans aksa göre eksantrik olarak tanımlanabilmektedir. Tez
çalışması kapsamında ortak yalıtım düzleminde bulunan çoklu yapıların
parametrik değişkenler altında yapı davranışları, yalnızca tek doğrultuda
yatay ötelenme serbestlik dereceleri dikkate alınarak incelenmiştir.
Bu sebeple hareket denklemleri, Şekil \ref{fig:multistructure}'de
sunulan örnek yapı üzerinden türetilecektir.

\section{Önceki Çalışmalar}

\begin{figure}[h]
\centering{}\includegraphics[width=1\linewidth]{TikZ/MultiStructure}
\caption{\label{fig:multistructure}Sismik yalıtımlı çoklu yapıların şekil
değiştirmiş hali.}
\end{figure}

Burada, $m$ kat kütlelerini, $c$ kat sönümünü, $k$ rijitlik değerini
ve $x$ ise yer değiştirmeleri belirtmektedir. Üst indiste büyüklüğün
göreli olarak yazıldığı konum belirtilmiştir. Yapı kat yer değiştirmelerinde
alt indislerde verilen ilk rakam yapı numarasını, ikinci rakam ise
büyüklüğün hangi kata ait olduğunu işaret etmektedir. Yalıtım düzlemi
yer değiştirmesi veya izolatör deplasmanları ise $x_{\text{b}}^{\text{g}}$
ile verilmiştir.



\chapter{DEPLASMAN ARTTIRMA KATSAYISI}

\label{CH3} 

Tez çalışması kapsamında sismik yalıtımlı yapıların doğrusal olmayan
hareket denklemlerinin yer ivmeleri etkisinde daha önce belirtilen
kapsam ve yöntemler dahilinde çözülebildiği MSBIS programı hazırlanmıştır.
Bu bölümde, oluşturulan algoritmanın doğrulaması yapılarak, yapısal
özellikleri açıklanmış ve ortak yalıtım düzleminde bulunan ikili yapıların,
parametrik incelenmesine dair prosedürler gösterilecektir. Daha sonra
hazırlanan çalışmaya ait sonuçlar paylaşılmıştır. Son olarak ikiden
fazla yapının ortak yalıtım düzleminde bulunması durumu için hazırlanan
örnek sistemler incelenecektir.

Numarasız denkleme örnek:

\section{Taşıyıcı Sistem Davranış Katsayısı}

Tez çalışması kapsamında sismik yalıtımlı yapıların doğrusal olmayan
hareket denklemlerinin yer ivmeleri etkisinde daha önce belirtilen
kapsam ve yöntemler dahilinde çözülebildiği MSBIS programı hazırlanmıştır.
Bu bölümde, oluşturulan algoritmanın doğrulaması yapılarak, yapısal
özellikleri açıklanmış ve ortak yalıtım düzleminde bulunan ikili yapıların,
parametrik incelenmesine dair prosedürler gösterilecektir. Daha sonra
hazırlanan çalışmaya ait sonuçlar paylaşılmıştır. Son olarak ikiden
fazla yapının ortak yalıtım düzleminde bulunması durumu için hazırlanan
örnek sistemler incelenecektir.

Numarasız denkleme örnek:
\[
\Delta=5x
\]


\section{Dayanım Fazlalığı Katsayısı}

\label{StructuralProperties} Tez çalışması kapsamında incelenen ortak
yalıtım düzleminde bulunan sismik yalıtımlı iki yapının değerlendirilmesi,
bu bölümde açıklanan yapı parametreleri kullanılarak gerçekleştirilmiştir.
Yapılar yalnızca yatay doğrusal kesme yaylarına sahip çok serbestlik
dereceli sistemlerden oluşmaktadır. Yapı elemanları, doğrusal elastik
olarak modellenmiştir. Dolayısıyla, üst yapı taşıyıcılarında deprem
yükleri altında oluşan kuvvet-yer değiştirme ilişkisinin doğrusal
elastik olduğu kabul edilmiştir. Şekil \ref{fig:structuralmodel}'de
belirtildiği gibi yalıtım düzlemi ve yapı kat döşemeleri düzlem içinde
rijit kabul edilerek tek bir kütle ile ifade edilmiştir. 
\begin{figure}[h!]
\centering{}\includegraphics{TikZ/StructuralModel} \caption{\label{fig:structuralmodel}3 nolu yapıya ait analitik model ve yapının
şekil değiştirmiş hali.}
 
\end{figure}

Yapının kat rijitlikleri, tüm kolonların yatay rijitliklerinin denklem
\ref{lateralstiffness}'de verilen formüle göre belirlenerek toplanması
ile elde edilmiştir. Her bir kat için yatay rijitlik değeri, toplam
16 adet 60 $\times$ 60 cm boyutlarında betonarme kolonların yatay
rijitlikleri toplamına eşit olarak alınmıştır. Tüm yapı ve katlar
için kat yüksekliği h=4m olarak belirlenmiştir. Betonarme elastisite
modulü $E_{\text{c}}=32000\text{ MPa}$ olarak kabul edilmiştir. 
\begin{equation}
\sum\limits _{i=1}^{n}k_{i}=\dfrac{12E_{\text{c}}I}{h^{3}}\label{lateralstiffness}
\end{equation}
Burada, $k$ her bir kolona ait yatay rijitliği, $E_{\text{c}}$ betonarme
elastisite modulünü, $I$ atalet momentini, h kat yüksekliğini ve
n bir katta bulunan kolon adedini ifade etmektedir. Sismik yalıtım
sistemi ise tüm izolatörlerin yapısal özelliklerinin toplamı ile ifade
edilen doğrusal olmayan yatay kesme yayı ile tanımlanmıştır. Ayrıca
sismik yalıtımlı yapının sahip olduğu sönüm, izolatörlerin doğrusal
olmayan kesme yaylarında sönümlediği enerji ve üst yapının sahip olduğu
Rayleigh sönümü olmak üzere ayrıklaştırılmıştır. Yapısal sönüm modellerine
ait detaylar Bölüm detaylı olarak açıklanmıştır.

\section{Süneklik Azaltma Katsayısı}

sadsadasdasdaşkms alsdöaşsd 

\begin{table}[h]
\centering{}\caption{\label{structures2}Çalışma kapsamında incelenen sistemlerin tabanı
ankastre olması durumu için hesaplanan yapısal özellikleri.}
\begin{tabular}{ccccc}
\hline 
Yapı \#  & Kat Adedi  & $T_{1}$  & $\omega_{1}$  & $\omega_{\text{n}}$ \tabularnewline
\hline 
1  & 1  & 0.1573  & 39.94  & - \tabularnewline
2  & 2  & 0.2546  & 24.68  & 64.62 \tabularnewline
3  & 3  & 0.3535  & 17.77  & 71.97 \tabularnewline
4  & 4  & 0.4530  & 13.87  & 75.06 \tabularnewline
5  & 5  & 0.5527  & 11.37  & 76.64 \tabularnewline
6  & 6  & 0.6526  & 9.63  & 77.56 \tabularnewline
7  & 7  & 0.7525  & 8.35  & 78.13 \tabularnewline
8  & 8  & 0.8525  & 7.37  & 78.52 \tabularnewline
9  & 9  & 0.9525  & 6.60  & 78.79 \tabularnewline
10  & 10  & 1.0526  & 5.97  & 78.98 \tabularnewline
\hline 
\end{tabular}
\end{table}

Tez çalışması kapsamında ortak yalıtım düzleminde bulunan yapıların,
dinamik özelliklerinin taban kesme kuvvetlerine olan etkisini incelemek
üzere on adet yapı seçilmiştir. Kat sayıları birden ona kadar değişen
yapıların kat rijitlikleri $\text{k}_{i}=0.0108\text{ m}^{4}$, kat
kütleleri $m_{i}=650\text{ ton}$ ve üst yapıya ait sönüm oranları
$\xi=0.05$ olmak üzere tümünde aynıdır. Kat adedine bağlı olarak
değişen yapı doğal periyotları ile birinci ve sonuncu açısal frekans
değerleri Çizelge \ref{structures2}'de sunulmuştur. Ayrıca, yalıtım
düzlemi kütlesi her bir yapı için $m_{\text{b}}=981\text{ ton}$ olarak
alınmıştır. Yalıtım birimlerine ait dinamik özellikler parametrik
olarak belirlenmiş olup Bölüm'de detaylı olarak açıklanmıştır.

\section{Deplasman Arttırma Katsayısı}

adsadad

\section{Eşit Yerdeğiştirme ve Eşit Enerji Prensipleri}

sdasdalsdkalms



\chapter{PARAMETRİK ÇALIŞMA}

\section{Parametrik Çalışmada Kullanılacak Yapılar}

\label{CH4} 

Bu çalışmada ortak yalıtım düzleminde bulunan sismik yalıtımlı iki
yapı için parametrik bir çalışma yapılmıştır. Analizler bağımsız veya
ortak yalıtım düzleminde bulunan sismik yalıtımlı yapıların doğrusal
olmayan dinamik analizlerinin parametrik olarak yapılabildiği MSBIS
programı yardımıyla gerçekleştirilmiştir. Üst yapı yapısal özelliklerinin
parametrik olarak değiştiği analizlerin tamamı, yalıtım birimlerinin
farklı eşdeğer periyot ve sönüm değerleri için tekrarlanmıştır. Burada
ortak yalıtım düzleminde bulunan sismik yalıtımlı birinci yapının
yapısal özellikleri değişen ikinci yapı ve izolatör parametreleri
değişimi ile meydana gelen etkileşimlerini incelemek amaçlanmıştır.
Buna göre birinci yapıda oluşan kat kesme kuvvetlerinde, göreli kat
ötelemelerinde ve kat ivmelerinde oluşan değişim irdelenmiştir. Taban
kesme kuvvetinin üst yapıya dağılımı ise yalıtım birimlerinin değişen
dinamik özelliklerine göre belirlenmiştir. Ayrıca ortak yalıtım düzlemindeki
iki yapı için gerekli deprem derz mesafeleri doğrusal yöntemler ile
bulunan sonuçlarla kıyaslanmıştır.

Çalışmada elde edilen sonuçlara göre yalıtım birimlerinin artan eşdeğer
sönüm oranına karşılık gelen büyük doğrusalsızlık sebebi ile ortak
yalıtım düzleminde bulunan yapıların yüksek mod etkilerinin göreli
yer değiştirme ve iç kuvvetlerinde önemli artışlara sebep olduğu görülmüştür.
Bu artışların iki yapının açısal frekanslarının ayrıklaşması ile arttığı
tespit edilmiştir. Dolayısıyla iki yapının aynı açısal frekansa sahip
olması durumunda elde edilen iç kuvvet ve yer değiştirmeler, yalıtım
birimlerinin aynı eşdeğer periyot ve sönüm değerleri için eşit bulunmaktadır.
Bununla birlikte yalıtım birimlerinin artan eşdeğer sönüm değerleri
için sismik yalıtımlı yapı taban kesme kuvvetlerinin üst katlara dağılımında
meydana gelen dikdörtgen form yerini üst katlarda daha büyük kuvvetlerin
oluştuğu ters üçgen formuna bıraktığı görülmüştür.

\section{Yükler ve Yük Kombinasyonları}

Tez çalışması kapsamında elde edilen sonuçlar aşağıda listelenmiştir. 
\begin{enumerate}
\item Yalıtım birimlerinin sahip olduğu histeretik sönüm nedeniyle meydana
gelen yüksek mod etkisi doğrusal analizler ile bulunan sonuçların
değişmesine neden olmaktadır.
\item Ortak yalıtım düzleminde bulunan ve farklı açısal frekanslara sahip
yapıların şekil değiştirme ve iç kuvvetleri, bağımsız yalıtım düzleminde
bulunan aynı yapılara kıyasla artış göstermektedir.
\item Yapı taban kesme kuvvetlerinin üst katlara dağılımının eşdeğer sönüm
oranının artışı ile dikdörtgen formdan ters üçgen formuna doğru değiştiği
görülmüştür. Ayrıca iki yapının açısal frekanslarına bağlı olarak
bu değişimin artış veya azalışta olduğu tespit edilmiştir.
\item Deprem derz mesafelerinde doğrusal yöntemler ile bulunan değerlerin
yeterli olduğu tespit edilmiştir.
\item Ortak yalıtım düzleminde bulunan 1. yapı taban kesme kuvveti katsayısının
toplam taban kesme kuvveti katsayısına göre bağıl hatasının en büyük
değeri, 1 ve 10 katlı iki yapının ortak yalıtım düzleminde bulunması
durumunda ve $T_{\text{eff}}=$ 2.5sn ve $\xi_{\text{eff}}$=0.3 değerleri
için 2.20 olarak hesaplanmıştır.
\item Ortak yalıtım düzleminde bulunan yapıların taban kesme kuvveti katsayılarının
toplam taban kesme kuvveti katsayısına göre bağıl hatasının en büyük
değeri, 2 ve 10 katlı iki yapının ortak yalıtım düzleminde bulunması
durumunda ve $T_{\text{eff}}=$ 2.5sn ve $\xi_{\text{eff}}$=0.3 değerleri
için 1.20 olarak hesaplanmıştır.
\item Yalıtım birimi kesme kuvveti katsayısının ortak yalıtım düzleminde
bulunan yapıların toplam taban kesme kuvveti katsayısına göre bağıl
hatasının en büyük değeri, 1 ve 5 katlı iki yapının ortak yalıtım
düzleminde bulunması durumunda ve $T_{\text{eff}}=$ 2.5sn ve $\xi_{\text{eff}}$=0.3
değerleri için -0.218 olarak hesaplanmıştır.
\item 1 ve 10 katlı iki yapının ortak yalıtım düzleminde bulunması durumunda
$T_{\text{eff}}=$ 4sn ve $\xi_{\text{eff}}$=0.3 değerleri için,
1 katlı yapının kesme kuvveti kat ivmesi ve göreli kat ötelemesinin,
bağımsız yalıtım düzleminde bulunması durumuna göre 3.27 katına çıktığı
belirlenmiştir.
\item 1 ve 10 katlı iki yapının ortak yalıtım düzleminde bulunması durumunda
10 katlı yapının kat kesme kuvvetleri, bağımsız yalıtım düzleminde
bulunması durumuna göre alt katlarda en büyük azalma $T_{\text{eff}}=$
1.5sn ve $\xi_{\text{eff}}$=0.3 değerleri için \%21.6 ve üst katta
ise en büyük azalma $T_{\text{eff}}=$ 4sn ve $\xi_{\text{eff}}$=0.2
değerleri için \%18.9 olarak tespit edilmiştir.
\item 10 katlı yapının ortak yalıtım düzleminde bulunması durumunda kat
kesme kuvvetleri, bağımsız yalıtım düzleminde bulunması durumuna göre
alt katlarda en fazla \%53.1 artış $T_{\text{eff}}=$ 1.5sn ve $\xi_{\text{eff}}$=0.3
değerleri için 6 katlı yapı ile birlikte bulunması durumu için gerçekleşmiştir.
Ayrıca üst katta en büyük azalma $T_{\text{eff}}=$ 4sn ve $\xi_{\text{eff}}$=0.2
değerleri için 2 katlı yapı ile birlikte bulunmaları durumu için \%26.1
olarak saptanmıştır.
\item 10 katlı yapının ortak yalıtım düzleminde bulunması durumunda göreli
kat ötelemeleri, bağımsız yalıtım düzleminde bulunması durumuna göre
alt katlarda en fazla \%53.9 artış $T_{\text{eff}}=$ 1.5sn ve $\xi_{\text{eff}}$=0.3
değerleri için 6 katlı yapı ile birlikte bulunması durumu için gerçekleşmiştir.
Ayrıca üst katta en büyük azalma $T_{\text{eff}}=$ 4sn ve $\xi_{\text{eff}}$=0.2
değerleri için 2 katlı yapı ile birlikte bulunmaları durumu için \%26.4
olarak saptanmıştır. 
\end{enumerate}

\section{Moment Dayanımlı Çelik Çerçeveli Sistemlerin Tasarımı}

Tez çalışması kapsamında elde edilen sonuçlar aşağıda listelenmiştir.
Tez çalışması kapsamında elde edilen sonuçlar aşağıda listelenmiştir.
Tez çalışması kapsamında elde edilen sonuçlar aşağıda listelenmiştir.
Tez çalışması kapsamında elde edilen sonuçlar aşağıda listelenmiştir.
Tez çalışması kapsamında elde edilen sonuçlar aşağıda listelenmiştir.
Tez çalışması kapsamında elde edilen sonuçlar aşağıda listelenmiştir.
Tez çalışması kapsamında elde edilen sonuçlar aşağıda listelenmiştir.
Tez çalışması kapsamında elde edilen sonuçlar aşağıda listelenmiştir.
Tez çalışması kapsamında elde edilen sonuçlar aşağıda listelenmiştir.
Tez çalışması kapsamında elde edilen sonuçlar aşağıda listelenmiştir.
Tez çalışması kapsamında elde edilen sonuçlar aşağıda listelenmiştir.
Tez çalışması kapsamında elde edilen sonuçlar aşağıda listelenmiştir.
Tez çalışması kapsamında elde edilen sonuçlar aşağıda listelenmiştir.
Tez çalışması kapsamında elde edilen sonuçlar aşağıda listelenmiştir.
Tez çalışması kapsamında elde edilen sonuçlar aşağıda listelenmiştir.
Tez çalışması kapsamında elde edilen sonuçlar aşağıda listelenmiştir.
Tez çalışması kapsamında elde edilen sonuçlar aşağıda listelenmiştir.
Tez çalışması kapsamında elde edilen sonuçlar aşağıda listelenmiştir.
Tez çalışması kapsamında elde edilen sonuçlar aşağıda listelenmiştir.
Tez çalışması kapsamında elde edilen sonuçlar aşağıda listelenmiştir.
Tez çalışması kapsamında elde edilen sonuçlar aşağıda listelenmiştir.
Tez çalışması kapsamında elde edilen sonuçlar aşağıda listelenmiştir.
Tez çalışması kapsamında elde edilen sonuçlar aşağıda listelenmiştir.
Tez çalışması kapsamında elde edilen sonuçlar aşağıda listelenmiştir.
Tez çalışması kapsamında elde edilen sonuçlar aşağıda listelenmiştir.
Tez çalışması kapsamında elde edilen sonuçlar aşağıda listelenmiştir.
Tez çalışması kapsamında elde edilen sonuçlar aşağıda listelenmiştir.
Tez çalışması kapsamında elde edilen sonuçlar aşağıda listelenmiştir.
Tez çalışması kapsamında elde edilen sonuçlar aşağıda listelenmiştir.
Tez çalışması kapsamında elde edilen sonuçlar aşağıda listelenmiştir.
Tez çalışması kapsamında elde edilen sonuçlar aşağıda listelenmiştir.
Tez çalışması kapsamında elde edilen sonuçlar aşağıda listelenmiştir.
Tez çalışması kapsamında elde edilen sonuçlar aşağıda listelenmiştir.
Tez çalışması kapsamında elde edilen sonuçlar aşağıda listelenmiştir.
Tez çalışması kapsamında elde edilen sonuçlar aşağıda listelenmiştir.
Tez çalışması kapsamında elde edilen sonuçlar aşağıda listelenmiştir.
Tez çalışması kapsamında elde edilen sonuçlar aşağıda listelenmiştir.
Tez çalışması kapsamında elde edilen sonuçlar aşağıda listelenmiştir.
Tez çalışması kapsamında elde edilen sonuçlar aşağıda listelenmiştir.
Tez çalışması kapsamında elde edilen sonuçlar aşağıda listelenmiştir.
Tez çalışması kapsamında elde edilen sonuçlar aşağıda listelenmiştir.
Tez çalışması kapsamında elde edilen sonuçlar aşağıda listelenmiştir.
Tez çalışması kapsamında elde edilen sonuçlar aşağıda listelenmiştir.
Tez çalışması kapsamında elde edilen sonuçlar aşağıda listelenmiştir.
Tez çalışması kapsamında elde edilen sonuçlar aşağıda listelenmiştir.
Tez çalışması kapsamında elde edilen sonuçlar aşağıda listelenmiştir.
Tez çalışması kapsamında elde edilen sonuçlar aşağıda listelenmiştir.
Tez çalışması kapsamında elde edilen sonuçlar aşağıda listelenmiştir.
Tez çalışması kapsamında elde edilen sonuçlar aşağıda listelenmiştir.
Tez çalışması kapsamında elde edilen sonuçlar aşağıda listelenmiştir.
Tez çalışması kapsamında elde edilen sonuçlar aşağıda listelenmiştir.
Tez çalışması kapsamında elde edilen sonuçlar aşağıda listelenmiştir.
Tez çalışması kapsamında elde edilen sonuçlar aşağıda listelenmiştir.
Tez çalışması kapsamında elde edilen sonuçlar aşağıda listelenmiştir.
Tez çalışması kapsamında elde edilen sonuçlar aşağıda listelenmiştir.
Tez çalışması kapsamında elde edilen sonuçlar aşağıda listelenmiştir.
Tez çalışması kapsamında elde edilen sonuçlar aşağıda listelenmiştir.
Tez çalışması kapsamında elde edilen sonuçlar aşağıda listelenmiştir.
Tez çalışması kapsamında elde edilen sonuçlar aşağıda listelenmiştir.
Tez çalışması kapsamında elde edilen sonuçlar aşağıda listelenmiştir.
Tez çalışması kapsamında elde edilen sonuçlar aşağıda listelenmiştir.
Tez çalışması kapsamında elde edilen sonuçlar aşağıda listelenmiştir.
Tez çalışması kapsamında elde edilen sonuçlar aşağıda listelenmiştir.
Tez çalışması kapsamında elde edilen sonuçlar aşağıda listelenmiştir.
Tez çalışması kapsamında elde edilen sonuçlar aşağıda listelenmiştir.
Tez çalışması kapsamında elde edilen sonuçlar aşağıda listelenmiştir.
Tez çalışması kapsamında elde edilen sonuçlar aşağıda listelenmiştir.
Tez çalışması kapsamında elde edilen sonuçlar aşağıda listelenmiştir.
Tez çalışması kapsamında elde edilen sonuçlar aşağıda listelenmiştir.
Tez çalışması kapsamında elde edilen sonuçlar aşağıda listelenmiştir.
Tez çalışması kapsamında elde edilen sonuçlar aşağıda listelenmiştir.
Tez çalışması kapsamında elde edilen sonuçlar aşağıda listelenmiştir.
Tez çalışması kapsamında elde edilen sonuçlar aşağıda listelenmiştir.
Tez çalışması kapsamında elde edilen sonuçlar aşağıda listelenmiştir.
Tez çalışması kapsamında elde edilen sonuçlar aşağıda listelenmiştir.
Tez çalışması kapsamında elde edilen sonuçlar aşağıda listelenmiştir.
Tez çalışması kapsamında elde edilen sonuçlar aşağıda listelenmiştir.
Tez çalışması kapsamında elde edilen sonuçlar aşağıda listelenmiştir.
Tez çalışması kapsamında elde edilen sonuçlar aşağıda listelenmiştir.
Tez çalışması kapsamında elde edilen sonuçlar aşağıda listelenmiştir.
Tez çalışması kapsamında elde edilen sonuçlar aşağıda listelenmiştir.
Tez çalışması kapsamında elde edilen sonuçlar aşağıda listelenmiştir.
Tez çalışması kapsamında elde edilen sonuçlar aşağıda listelenmiştir.
Tez çalışması kapsamında elde edilen sonuçlar aşağıda listelenmiştir.
Tez çalışması kapsamında elde edilen sonuçlar aşağıda listelenmiştir.
Tez çalışması kapsamında elde edilen sonuçlar aşağıda listelenmiştir.
Tez çalışması kapsamında elde edilen sonuçlar aşağıda listelenmiştir.
Tez çalışması kapsamında elde edilen sonuçlar aşağıda listelenmiştir.
Tez çalışması kapsamında elde edilen sonuçlar aşağıda listelenmiştir.
Tez çalışması kapsamında elde edilen sonuçlar aşağıda listelenmiştir.
Tez çalışması kapsamında elde edilen sonuçlar aşağıda listelenmiştir.
Tez çalışması kapsamında elde edilen sonuçlar aşağıda listelenmiştir.
Tez çalışması kapsamında elde edilen sonuçlar aşağıda listelenmiştir.
Tez çalışması kapsamında elde edilen sonuçlar aşağıda listelenmiştir.
Tez çalışması kapsamında elde edilen sonuçlar aşağıda listelenmiştir.
Tez çalışması kapsamında elde edilen sonuçlar aşağıda listelenmiştir.
Tez çalışması kapsamında elde edilen sonuçlar aşağıda listelenmiştir.
Tez çalışması kapsamında elde edilen sonuçlar aşağıda listelenmiştir.
Tez çalışması kapsamında elde edilen sonuçlar aşağıda listelenmiştir.
Tez çalışması kapsamında elde edilen sonuçlar aşağıda listelenmiştir.
Tez çalışması kapsamında elde edilen sonuçlar aşağıda listelenmiştir.
Tez çalışması kapsamında elde edilen sonuçlar aşağıda listelenmiştir.
Tez çalışması kapsamında elde edilen sonuçlar aşağıda listelenmiştir.
Tez çalışması kapsamında elde edilen sonuçlar aşağıda listelenmiştir.
Tez çalışması kapsamında elde edilen sonuçlar aşağıda listelenmiştir.
Tez çalışması kapsamında elde edilen sonuçlar aşağıda listelenmiştir.
Tez çalışması kapsamında elde edilen sonuçlar aşağıda listelenmiştir.
Tez çalışması kapsamında elde edilen sonuçlar aşağıda listelenmiştir.
Tez çalışması kapsamında elde edilen sonuçlar aşağıda listelenmiştir.
Tez çalışması kapsamında elde edilen sonuçlar aşağıda listelenmiştir.
Tez çalışması kapsamında elde edilen sonuçlar aşağıda listelenmiştir.
Tez çalışması kapsamında elde edilen sonuçlar aşağıda listelenmiştir.
Tez çalışması kapsamında elde edilen sonuçlar aşağıda listelenmiştir.
Tez çalışması kapsamında elde edilen sonuçlar aşağıda listelenmiştir.
Tez çalışması kapsamında elde edilen sonuçlar aşağıda listelenmiştir.
Tez çalışması kapsamında elde edilen sonuçlar aşağıda listelenmiştir.
Tez çalışması kapsamında elde edilen sonuçlar aşağıda listelenmiştir.
Tez çalışması kapsamında elde edilen sonuçlar aşağıda listelenmiştir.
Tez çalışması kapsamında elde edilen sonuçlar aşağıda listelenmiştir.
Tez çalışması kapsamında elde edilen sonuçlar aşağıda listelenmiştir.
Tez çalışması kapsamında elde edilen sonuçlar aşağıda listelenmiştir.
Tez çalışması kapsamında elde edilen sonuçlar aşağıda listelenmiştir.
Tez çalışması kapsamında elde edilen sonuçlar aşağıda listelenmiştir.
Tez çalışması kapsamında elde edilen sonuçlar aşağıda listelenmiştir.
Tez çalışması kapsamında elde edilen sonuçlar aşağıda listelenmiştir.
Tez çalışması kapsamında elde edilen sonuçlar aşağıda listelenmiştir.
Tez çalışması kapsamında elde edilen sonuçlar aşağıda listelenmiştir.
Tez çalışması kapsamında elde edilen sonuçlar aşağıda listelenmiştir.
Tez çalışması kapsamında elde edilen sonuçlar aşağıda listelenmiştir.
Tez çalışması kapsamında elde edilen sonuçlar aşağıda listelenmiştir.
Tez çalışması kapsamında elde edilen sonuçlar aşağıda listelenmiştir.
Tez çalışması kapsamında elde edilen sonuçlar aşağıda listelenmiştir.
Tez çalışması kapsamında elde edilen sonuçlar aşağıda listelenmiştir.
Tez çalışması kapsamında elde edilen sonuçlar aşağıda listelenmiştir.
Tez çalışması kapsamında elde edilen sonuçlar aşağıda listelenmiştir.
Tez çalışması kapsamında elde edilen sonuçlar aşağıda listelenmiştir.
Tez çalışması kapsamında elde edilen sonuçlar aşağıda listelenmiştir.
Tez çalışması kapsamında elde edilen sonuçlar aşağıda listelenmiştir.
Tez çalışması kapsamında elde edilen sonuçlar aşağıda listelenmiştir.
Tez çalışması kapsamında elde edilen sonuçlar aşağıda listelenmiştir.
Tez çalışması kapsamında elde edilen sonuçlar aşağıda listelenmiştir.
Tez çalışması kapsamında elde edilen sonuçlar aşağıda listelenmiştir.
Tez çalışması kapsamında elde edilen sonuçlar aşağıda listelenmiştir.
Tez çalışması kapsamında elde edilen sonuçlar aşağıda listelenmiştir.
Tez çalışması kapsamında elde edilen sonuçlar aşağıda listelenmiştir.
Tez çalışması kapsamında elde edilen sonuçlar aşağıda listelenmiştir.
Tez çalışması kapsamında elde edilen sonuçlar aşağıda listelenmiştir.
Tez çalışması kapsamında elde edilen sonuçlar aşağıda listelenmiştir.
Tez çalışması kapsamında elde edilen sonuçlar aşağıda listelenmiştir.
Tez çalışması kapsamında elde edilen sonuçlar aşağıda listelenmiştir.
Tez çalışması kapsamında elde edilen sonuçlar aşağıda listelenmiştir.
Tez çalışması kapsamında elde edilen sonuçlar aşağıda listelenmiştir.
Tez çalışması kapsamında elde edilen sonuçlar aşağıda listelenmiştir.
Tez çalışması kapsamında elde edilen sonuçlar aşağıda listelenmiştir.
Tez çalışması kapsamında elde edilen sonuçlar aşağıda listelenmiştir.
Tez çalışması kapsamında elde edilen sonuçlar aşağıda listelenmiştir.
Tez çalışması kapsamında elde edilen sonuçlar aşağıda listelenmiştir.
Tez çalışması kapsamında elde edilen sonuçlar aşağıda listelenmiştir.
Tez çalışması kapsamında elde edilen sonuçlar aşağıda listelenmiştir.
Tez çalışması kapsamında elde edilen sonuçlar aşağıda listelenmiştir.
Tez çalışması kapsamında elde edilen sonuçlar aşağıda listelenmiştir.
Tez çalışması kapsamında elde edilen sonuçlar aşağıda listelenmiştir.
Tez çalışması kapsamında elde edilen sonuçlar aşağıda listelenmiştir.
Tez çalışması kapsamında elde edilen sonuçlar aşağıda listelenmiştir.
Tez çalışması kapsamında elde edilen sonuçlar aşağıda listelenmiştir.
Tez çalışması kapsamında elde edilen sonuçlar aşağıda listelenmiştir.
Tez çalışması kapsamında elde edilen sonuçlar aşağıda listelenmiştir.
Tez çalışması kapsamında elde edilen sonuçlar aşağıda listelenmiştir.
Tez çalışması kapsamında elde edilen sonuçlar aşağıda listelenmiştir.
Tez çalışması kapsamında elde edilen sonuçlar aşağıda listelenmiştir.
Tez çalışması kapsamında elde edilen sonuçlar aşağıda listelenmiştir.
Tez çalışması kapsamında elde edilen sonuçlar aşağıda listelenmiştir.
Tez çalışması kapsamında elde edilen sonuçlar aşağıda listelenmiştir.
Tez çalışması kapsamında elde edilen sonuçlar aşağıda listelenmiştir.
Tez çalışması kapsamında elde edilen sonuçlar aşağıda listelenmiştir.
Tez çalışması kapsamında elde edilen sonuçlar aşağıda listelenmiştir.
Tez çalışması kapsamında elde edilen sonuçlar aşağıda listelenmiştir.
Tez çalışması kapsamında elde edilen sonuçlar aşağıda listelenmiştir.
Tez çalışması kapsamında elde edilen sonuçlar aşağıda listelenmiştir.
Tez çalışması kapsamında elde edilen sonuçlar aşağıda listelenmiştir.
Tez çalışması kapsamında elde edilen sonuçlar aşağıda listelenmiştir.
Tez çalışması kapsamında elde edilen sonuçlar aşağıda listelenmiştir.
Tez çalışması kapsamında elde edilen sonuçlar aşağıda listelenmiştir.
Tez çalışması kapsamında elde edilen sonuçlar aşağıda listelenmiştir.
Tez çalışması kapsamında elde edilen sonuçlar aşağıda listelenmiştir.
Tez çalışması kapsamında elde edilen sonuçlar aşağıda listelenmiştir.
Tez çalışması kapsamında elde edilen sonuçlar aşağıda listelenmiştir.
Tez çalışması kapsamında elde edilen sonuçlar aşağıda listelenmiştir.
Tez çalışması kapsamında elde edilen sonuçlar aşağıda listelenmiştir.
Tez çalışması kapsamında elde edilen sonuçlar aşağıda listelenmiştir.
Tez çalışması kapsamında elde edilen sonuçlar aşağıda listelenmiştir.
Tez çalışması kapsamında elde edilen sonuçlar aşağıda listelenmiştir.
Tez çalışması kapsamında elde edilen sonuçlar aşağıda listelenmiştir.
Tez çalışması kapsamında elde edilen sonuçlar aşağıda listelenmiştir.
Tez çalışması kapsamında elde edilen sonuçlar aşağıda listelenmiştir.
Tez çalışması kapsamında elde edilen sonuçlar aşağıda listelenmiştir.
Tez çalışması kapsamında elde edilen sonuçlar aşağıda listelenmiştir.
Tez çalışması kapsamında elde edilen sonuçlar aşağıda listelenmiştir.
Tez çalışması kapsamında elde edilen sonuçlar aşağıda listelenmiştir.
Tez çalışması kapsamında elde edilen sonuçlar aşağıda listelenmiştir.
Tez çalışması kapsamında elde edilen sonuçlar aşağıda listelenmiştir.
Tez çalışması kapsamında elde edilen sonuçlar aşağıda listelenmiştir.
Tez çalışması kapsamında elde edilen sonuçlar aşağıda listelenmiştir.
Tez çalışması kapsamında elde edilen sonuçlar aşağıda listelenmiştir.
Tez çalışması kapsamında elde edilen sonuçlar aşağıda listelenmiştir.
Tez çalışması kapsamında elde edilen sonuçlar aşağıda listelenmiştir.
Tez çalışması kapsamında elde edilen sonuçlar aşağıda listelenmiştir.
Tez çalışması kapsamında elde edilen sonuçlar aşağıda listelenmiştir.
Tez çalışması kapsamında elde edilen sonuçlar aşağıda listelenmiştir.
Tez çalışması kapsamında elde edilen sonuçlar aşağıda listelenmiştir.
Tez çalışması kapsamında elde edilen sonuçlar aşağıda listelenmiştir.
Tez çalışması kapsamında elde edilen sonuçlar aşağıda listelenmiştir.
Tez çalışması kapsamında elde edilen sonuçlar aşağıda listelenmiştir.
Tez çalışması kapsamında elde edilen sonuçlar aşağıda listelenmiştir.
Tez çalışması kapsamında elde edilen sonuçlar aşağıda listelenmiştir.
Tez çalışması kapsamında elde edilen sonuçlar aşağıda listelenmiştir.
Tez çalışması kapsamında elde edilen sonuçlar aşağıda listelenmiştir.
Tez çalışması kapsamında elde edilen sonuçlar aşağıda listelenmiştir.
Tez çalışması kapsamında elde edilen sonuçlar aşağıda listelenmiştir.
Tez çalışması kapsamında elde edilen sonuçlar aşağıda listelenmiştir.
Tez çalışması kapsamında elde edilen sonuçlar aşağıda listelenmiştir.
Tez çalışması kapsamında elde edilen sonuçlar aşağıda listelenmiştir.
Tez çalışması kapsamında elde edilen sonuçlar aşağıda listelenmiştir.
Tez çalışması kapsamında elde edilen sonuçlar aşağıda listelenmiştir.
Tez çalışması kapsamında elde edilen sonuçlar aşağıda listelenmiştir.
Tez çalışması kapsamında elde edilen sonuçlar aşağıda listelenmiştir.
Tez çalışması kapsamında elde edilen sonuçlar aşağıda listelenmiştir.
Tez çalışması kapsamında elde edilen sonuçlar aşağıda listelenmiştir.
Tez çalışması kapsamında elde edilen sonuçlar aşağıda listelenmiştir.
Tez çalışması kapsamında elde edilen sonuçlar aşağıda listelenmiştir.
Tez çalışması kapsamında elde edilen sonuçlar aşağıda listelenmiştir.
Tez çalışması kapsamında elde edilen sonuçlar aşağıda listelenmiştir.
Tez çalışması kapsamında elde edilen sonuçlar aşağıda listelenmiştir.
Tez çalışması kapsamında elde edilen sonuçlar aşağıda listelenmiştir.
Tez çalışması kapsamında elde edilen sonuçlar aşağıda listelenmiştir.
Tez çalışması kapsamında elde edilen sonuçlar aşağıda listelenmiştir.
Tez çalışması kapsamında elde edilen sonuçlar aşağıda listelenmiştir.
Tez çalışması kapsamında elde edilen sonuçlar aşağıda listelenmiştir.
Tez çalışması kapsamında elde edilen sonuçlar aşağıda listelenmiştir.
Tez çalışması kapsamında elde edilen sonuçlar aşağıda listelenmiştir.
Tez çalışması kapsamında elde edilen sonuçlar aşağıda listelenmiştir.
Tez çalışması kapsamında elde edilen sonuçlar aşağıda listelenmiştir.
Tez çalışması kapsamında elde edilen sonuçlar aşağıda listelenmiştir.
Tez çalışması kapsamında elde edilen sonuçlar aşağıda listelenmiştir.
Tez çalışması kapsamında elde edilen sonuçlar aşağıda listelenmiştir.
Tez çalışması kapsamında elde edilen sonuçlar aşağıda listelenmiştir.
Tez çalışması kapsamında elde edilen sonuçlar aşağıda listelenmiştir.
Tez çalışması kapsamında elde edilen sonuçlar aşağıda listelenmiştir.
Tez çalışması kapsamında elde edilen sonuçlar aşağıda listelenmiştir. 

\section{Birleşim Dönme Rijitliklerinin Seçimi}

Tez çalışması kapsamında elde edilen sonuçlar aşağıda listelenmiştir.
Tez çalışması kapsamında elde edilen sonuçlar aşağıda listelenmiştir.
Tez çalışması kapsamında elde edilen sonuçlar aşağıda listelenmiştir.
Tez çalışması kapsamında elde edilen sonuçlar aşağıda listelenmiştir.
Tez çalışması kapsamında elde edilen sonuçlar aşağıda listelenmiştir. 

Tez çalışması kapsamında elde edilen sonuçlar aşağıda listelenmiştir.
Tez çalışması kapsamında elde edilen sonuçlar aşağıda listelenmiştir.
Tez çalışması kapsamında elde edilen sonuçlar aşağıda listelenmiştir.
Tez çalışması kapsamında elde edilen sonuçlar aşağıda listelenmiştir.
Tez çalışması kapsamında elde edilen sonuçlar aşağıda listelenmiştir.
Tez çalışması kapsamında elde edilen sonuçlar aşağıda listelenmiştir.
Tez çalışması kapsamında elde edilen sonuçlar aşağıda listelenmiştir. 



\chapter{RİJİT VE YARI--RİJİT ÇERÇEVELERİN DEPLASMAN ARTTIRMA KATSAYISININ
BELİRLENMESİ}

\section{Doğrusal Olmayan Statik İtme Analizi}

Çalışmada elde edilen sonuçlara göre yalıtım birimlerinin artan eşdeğer
sönüm oranına karşılık gelen büyük doğrusalsızlık sebebi ile ortak
yalıtım düzleminde bulunan yapıların yüksek mod etkilerinin göreli
yer değiştirme ve iç kuvvetlerinde önemli artışlara sebep olduğu görülmüştür.
Bu artışların iki yapının açısal frekanslarının ayrıklaşması ile arttığı
tespit edilmiştir. Dolayısıyla iki yapının aynı açısal frekansa sahip
olması durumunda elde edilen iç kuvvet ve yer değiştirmeler, yalıtım
birimlerinin aynı eşdeğer periyot ve sönüm değerleri için eşit bulunmaktadır.
Bununla birlikte yalıtım birimlerinin artan eşdeğer sönüm değerleri
için sismik yalıtımlı yapı taban kesme kuvvetlerinin üst katlara dağılımında
meydana gelen dikdörtgen form yerini üst katlarda daha büyük kuvvetlerin
oluştuğu ters üçgen formuna bıraktığı görülmüştür.

Tez çalışması kapsamında elde edilen sonuçlar aşağıda listelenmiştir. 

\subsection{Kapasite Spektrumu Metodu}

Çalışmada elde edilen sonuçlara göre yalıtım birimlerinin artan eşdeğer
sönüm oranına karşılık gelen büyük doğrusalsızlık sebebi ile ortak
yalıtım düzleminde bulunan yapıların yüksek mod etkilerinin göreli
yer değiştirme ve iç kuvvetlerinde önemli artışlara sebep olduğu görülmüştür.
Bu artışların iki yapının açısal frekanslarının ayrıklaşması ile arttığı
tespit edilmiştir. Dolayısıyla iki yapının aynı açısal frekansa sahip
olması durumunda elde edilen iç kuvvet ve yer değiştirmeler, yalıtım
birimlerinin aynı eşdeğer periyot ve sönüm değerleri için eşit bulunmaktadır.
Bununla birlikte yalıtım birimlerinin artan eşdeğer sönüm değerleri
için sismik yalıtımlı yapı taban kesme kuvvetlerinin üst katlara dağılımında
meydana gelen dikdörtgen form yerini üst katlarda daha büyük kuvvetlerin
oluştuğu ters üçgen formuna bıraktığı görülmüştür.

Tez çalışması kapsamında elde edilen sonuçlar aşağıda listelenmiştir. 

Çalışmada elde edilen sonuçlara göre yalıtım birimlerinin artan eşdeğer
sönüm oranına karşılık gelen büyük doğrusalsızlık sebebi ile ortak
yalıtım düzleminde bulunan yapıların yüksek mod etkilerinin göreli
yer değiştirme ve iç kuvvetlerinde önemli artışlara sebep olduğu görülmüştür.
Bu artışların iki yapının açısal frekanslarının ayrıklaşması ile arttığı
tespit edilmiştir. Dolayısıyla iki yapının aynı açısal frekansa sahip
olması durumunda elde edilen iç kuvvet ve yer değiştirmeler, yalıtım
birimlerinin aynı eşdeğer periyot ve sönüm değerleri için eşit bulunmaktadır.
Bununla birlikte yalıtım birimlerinin artan eşdeğer sönüm değerleri
için sismik yalıtımlı yapı taban kesme kuvvetlerinin üst katlara dağılımında
meydana gelen dikdörtgen form yerini üst katlarda daha büyük kuvvetlerin
oluştuğu ters üçgen formuna bıraktığı görülmüştür.

Tez çalışması kapsamında elde edilen sonuçlar aşağıda listelenmiştir. 

\section{Zaman Tanım Alanında Doğrusal Olmayan Dinamik Analiz}

Çalışmada elde edilen sonuçlara göre yalıtım birimlerinin artan eşdeğer
sönüm oranına karşılık gelen büyük doğrusalsızlık sebebi ile ortak
yalıtım düzleminde bulunan yapıların yüksek mod etkilerinin göreli
yer değiştirme ve iç kuvvetlerinde önemli artışlara sebep olduğu görülmüştür.
Bu artışların iki yapının açısal frekanslarının ayrıklaşması ile arttığı
tespit edilmiştir. Dolayısıyla iki yapının aynı açısal frekansa sahip
olması durumunda elde edilen iç kuvvet ve yer değiştirmeler, yalıtım
birimlerinin aynı eşdeğer periyot ve sönüm değerleri için eşit bulunmaktadır.
Bununla birlikte yalıtım birimlerinin artan eşdeğer sönüm değerleri
için sismik yalıtımlı yapı taban kesme kuvvetlerinin üst katlara dağılımında
meydana gelen dikdörtgen form yerini üst katlarda daha büyük kuvvetlerin
oluştuğu ters üçgen formuna bıraktığı görülmüştür.

Tez çalışması kapsamında elde edilen sonuçlar aşağıda listelenmiştir. 

\subsection{Deprem Kayıtlarının Seçimi ve Ölçeklendirilmesi}

Çalışmada elde edilen sonuçlara göre yalıtım birimlerinin artan eşdeğer
sönüm oranına karşılık gelen büyük doğrusalsızlık sebebi ile ortak
yalıtım düzleminde bulunan yapıların yüksek mod etkilerinin göreli
yer değiştirme ve iç kuvvetlerinde önemli artışlara sebep olduğu görülmüştür.
Bu artışların iki yapının açısal frekanslarının ayrıklaşması ile arttığı
tespit edilmiştir. Dolayısıyla iki yapının aynı açısal frekansa sahip
olması durumunda elde edilen iç kuvvet ve yer değiştirmeler, yalıtım
birimlerinin aynı eşdeğer periyot ve sönüm değerleri için eşit bulunmaktadır.
Bununla birlikte yalıtım birimlerinin artan eşdeğer sönüm değerleri
için sismik yalıtımlı yapı taban kesme kuvvetlerinin üst katlara dağılımında
meydana gelen dikdörtgen form yerini üst katlarda daha büyük kuvvetlerin
oluştuğu ters üçgen formuna bıraktığı görülmüştür.

Tez çalışması kapsamında elde edilen sonuçlar aşağıda listelenmiştir. 



\chapter{ANALİZ SONUÇLARI}

\section{3 Katlı Yapıların Doğrusal Olmayan Analiz Sonuçları}

Çalışmada elde edilen sonuçlara göre yalıtım birimlerinin artan eşdeğer
sönüm oranına karşılık gelen büyük doğrusalsızlık sebebi ile ortak
yalıtım düzleminde bulunan yapıların yüksek mod etkilerinin göreli
yer değiştirme ve iç kuvvetlerinde önemli artışlara sebep olduğu görülmüştür.
Bu artışların iki yapının açısal frekanslarının ayrıklaşması ile arttığı
tespit edilmiştir. Dolayısıyla iki yapının aynı açısal frekansa sahip
olması durumunda elde edilen iç kuvvet ve yer değiştirmeler, yalıtım
birimlerinin aynı eşdeğer periyot ve sönüm değerleri için eşit bulunmaktadır.
Bununla birlikte yalıtım birimlerinin artan eşdeğer sönüm değerleri
için sismik yalıtımlı yapı taban kesme kuvvetlerinin üst katlara dağılımında
meydana gelen dikdörtgen form yerini üst katlarda daha büyük kuvvetlerin
oluştuğu ters üçgen formuna bıraktığı görülmüştür.

Tez çalışması kapsamında elde edilen sonuçlar aşağıda listelenmiştir. 

\subsection{Doğrusal Olmayan Statik Analizi Sonuçları}

Çalışmada elde edilen sonuçlara göre yalıtım birimlerinin artan eşdeğer
sönüm oranına karşılık gelen büyük doğrusalsızlık sebebi ile ortak
yalıtım düzleminde bulunan yapıların yüksek mod etkilerinin göreli
yer değiştirme ve iç kuvvetlerinde önemli artışlara sebep olduğu görülmüştür.
Bu artışların iki yapının açısal frekanslarının ayrıklaşması ile arttığı
tespit edilmiştir. Dolayısıyla iki yapının aynı açısal frekansa sahip
olması durumunda elde edilen iç kuvvet ve yer değiştirmeler, yalıtım
birimlerinin aynı eşdeğer periyot ve sönüm değerleri için eşit bulunmaktadır.
Bununla birlikte yalıtım birimlerinin artan eşdeğer sönüm değerleri
için sismik yalıtımlı yapı taban kesme kuvvetlerinin üst katlara dağılımında
meydana gelen dikdörtgen form yerini üst katlarda daha büyük kuvvetlerin
oluştuğu ters üçgen formuna bıraktığı görülmüştür.

\subsection{Doğrusal Olmayan Dinamik Analizi Sonuçları}

Tez çalışması kapsamında elde edilen sonuçlar aşağıda listelenmiştir. 

Çalışmada elde edilen sonuçlara göre yalıtım birimlerinin artan eşdeğer
sönüm oranına karşılık gelen büyük doğrusalsızlık sebebi ile ortak
yalıtım düzleminde bulunan yapıların yüksek mod etkilerinin göreli
yer değiştirme ve iç kuvvetlerinde önemli artışlara sebep olduğu görülmüştür.
Bu artışların iki yapının açısal frekanslarının ayrıklaşması ile arttığı
tespit edilmiştir. Dolayısıyla iki yapının aynı açısal frekansa sahip
olması durumunda elde edilen iç kuvvet ve yer değiştirmeler, yalıtım
birimlerinin aynı eşdeğer periyot ve sönüm değerleri için eşit bulunmaktadır.
Bununla birlikte yalıtım birimlerinin artan eşdeğer sönüm değerleri
için sismik yalıtımlı yapı taban kesme kuvvetlerinin üst katlara dağılımında
meydana gelen dikdörtgen form yerini üst katlarda daha büyük kuvvetlerin
oluştuğu ters üçgen formuna bıraktığı görülmüştür.

Tez çalışması kapsamında elde edilen sonuçlar aşağıda listelenmiştir. 

\section{9 Katlı Yapıların Doğrusal Olmayan Analiz Sonuçları}

Tez çalışması kapsamında elde edilen sonuçlar aşağıda listelenmiştir. 

Çalışmada elde edilen sonuçlara göre yalıtım birimlerinin artan eşdeğer
sönüm oranına karşılık gelen büyük doğrusalsızlık sebebi ile ortak
yalıtım düzleminde bulunan yapıların yüksek mod etkilerinin göreli
yer değiştirme ve iç kuvvetlerinde önemli artışlara sebep olduğu görülmüştür.
Bu artışların iki yapının açısal frekanslarının ayrıklaşması ile arttığı
tespit edilmiştir. Dolayısıyla iki yapının aynı açısal frekansa sahip
olması durumunda elde edilen iç kuvvet ve yer değiştirmeler, yalıtım
birimlerinin aynı eşdeğer periyot ve sönüm değerleri için eşit bulunmaktadır.
Bununla birlikte yalıtım birimlerinin artan eşdeğer sönüm değerleri
için sismik yalıtımlı yapı taban kesme kuvvetlerinin üst katlara dağılımında
meydana gelen dikdörtgen form yerini üst katlarda daha büyük kuvvetlerin
oluştuğu ters üçgen formuna bıraktığı görülmüştür.

Tez çalışması kapsamında elde edilen sonuçlar aşağıda listelenmiştir. 

\subsection{Doğrusal Olmayan Statik Analizi Sonuçları}

Tez çalışması kapsamında elde edilen sonuçlar aşağıda listelenmiştir. 

Çalışmada elde edilen sonuçlara göre yalıtım birimlerinin artan eşdeğer
sönüm oranına karşılık gelen büyük doğrusalsızlık sebebi ile ortak
yalıtım düzleminde bulunan yapıların yüksek mod etkilerinin göreli
yer değiştirme ve iç kuvvetlerinde önemli artışlara sebep olduğu görülmüştür.
Bu artışların iki yapının açısal frekanslarının ayrıklaşması ile arttığı
tespit edilmiştir. Dolayısıyla iki yapının aynı açısal frekansa sahip
olması durumunda elde edilen iç kuvvet ve yer değiştirmeler, yalıtım
birimlerinin aynı eşdeğer periyot ve sönüm değerleri için eşit bulunmaktadır.
Bununla birlikte yalıtım birimlerinin artan eşdeğer sönüm değerleri
için sismik yalıtımlı yapı taban kesme kuvvetlerinin üst katlara dağılımında
meydana gelen dikdörtgen form yerini üst katlarda daha büyük kuvvetlerin
oluştuğu ters üçgen formuna bıraktığı görülmüştür.

Tez çalışması kapsamında elde edilen sonuçlar aşağıda listelenmiştir. 

\subsection{Doğrusal Olmayan Dinamik Analizi Sonuçları}

Tez çalışması kapsamında elde edilen sonuçlar aşağıda listelenmiştir. 

Çalışmada elde edilen sonuçlara göre yalıtım birimlerinin artan eşdeğer
sönüm oranına karşılık gelen büyük doğrusalsızlık sebebi ile ortak
yalıtım düzleminde bulunan yapıların yüksek mod etkilerinin göreli
yer değiştirme ve iç kuvvetlerinde önemli artışlara sebep olduğu görülmüştür.
Bu artışların iki yapının açısal frekanslarının ayrıklaşması ile arttığı
tespit edilmiştir. Dolayısıyla iki yapının aynı açısal frekansa sahip
olması durumunda elde edilen iç kuvvet ve yer değiştirmeler, yalıtım
birimlerinin aynı eşdeğer periyot ve sönüm değerleri için eşit bulunmaktadır.
Bununla birlikte yalıtım birimlerinin artan eşdeğer sönüm değerleri
için sismik yalıtımlı yapı taban kesme kuvvetlerinin üst katlara dağılımında
meydana gelen dikdörtgen form yerini üst katlarda daha büyük kuvvetlerin
oluştuğu ters üçgen formuna bıraktığı görülmüştür.

Tez çalışması kapsamında elde edilen sonuçlar aşağıda listelenmiştir. 

\section{20 Katlı Yapıların Doğrusal Olmayan Analiz Sonuçları}

Tez çalışması kapsamında elde edilen sonuçlar aşağıda listelenmiştir. 

Çalışmada elde edilen sonuçlara göre yalıtım birimlerinin artan eşdeğer
sönüm oranına karşılık gelen büyük doğrusalsızlık sebebi ile ortak
yalıtım düzleminde bulunan yapıların yüksek mod etkilerinin göreli
yer değiştirme ve iç kuvvetlerinde önemli artışlara sebep olduğu görülmüştür.
Bu artışların iki yapının açısal frekanslarının ayrıklaşması ile arttığı
tespit edilmiştir. Dolayısıyla iki yapının aynı açısal frekansa sahip
olması durumunda elde edilen iç kuvvet ve yer değiştirmeler, yalıtım
birimlerinin aynı eşdeğer periyot ve sönüm değerleri için eşit bulunmaktadır.
Bununla birlikte yalıtım birimlerinin artan eşdeğer sönüm değerleri
için sismik yalıtımlı yapı taban kesme kuvvetlerinin üst katlara dağılımında
meydana gelen dikdörtgen form yerini üst katlarda daha büyük kuvvetlerin
oluştuğu ters üçgen formuna bıraktığı görülmüştür.

Tez çalışması kapsamında elde edilen sonuçlar aşağıda listelenmiştir. 

\subsection{Doğrusal Olmayan Statik Analiz Sonuçları}

Tez çalışması kapsamında elde edilen sonuçlar aşağıda listelenmiştir. 

Çalışmada elde edilen sonuçlara göre yalıtım birimlerinin artan eşdeğer
sönüm oranına karşılık gelen büyük doğrusalsızlık sebebi ile ortak
yalıtım düzleminde bulunan yapıların yüksek mod etkilerinin göreli
yer değiştirme ve iç kuvvetlerinde önemli artışlara sebep olduğu görülmüştür.
Bu artışların iki yapının açısal frekanslarının ayrıklaşması ile arttığı
tespit edilmiştir. Dolayısıyla iki yapının aynı açısal frekansa sahip
olması durumunda elde edilen iç kuvvet ve yer değiştirmeler, yalıtım
birimlerinin aynı eşdeğer periyot ve sönüm değerleri için eşit bulunmaktadır.
Bununla birlikte yalıtım birimlerinin artan eşdeğer sönüm değerleri
için sismik yalıtımlı yapı taban kesme kuvvetlerinin üst katlara dağılımında
meydana gelen dikdörtgen form yerini üst katlarda daha büyük kuvvetlerin
oluştuğu ters üçgen formuna bıraktığı görülmüştür.

Tez çalışması kapsamında elde edilen sonuçlar aşağıda listelenmiştir. 

\subsection{Doğrusal Olmayan Dinamik Analiz Sonuçları}

Tez çalışması kapsamında elde edilen sonuçlar aşağıda listelenmiştir. 

Çalışmada elde edilen sonuçlara göre yalıtım birimlerinin artan eşdeğer
sönüm oranına karşılık gelen büyük doğrusalsızlık sebebi ile ortak
yalıtım düzleminde bulunan yapıların yüksek mod etkilerinin göreli
yer değiştirme ve iç kuvvetlerinde önemli artışlara sebep olduğu görülmüştür.
Bu artışların iki yapının açısal frekanslarının ayrıklaşması ile arttığı
tespit edilmiştir. Dolayısıyla iki yapının aynı açısal frekansa sahip
olması durumunda elde edilen iç kuvvet ve yer değiştirmeler, yalıtım
birimlerinin aynı eşdeğer periyot ve sönüm değerleri için eşit bulunmaktadır.
Bununla birlikte yalıtım birimlerinin artan eşdeğer sönüm değerleri
için sismik yalıtımlı yapı taban kesme kuvvetlerinin üst katlara dağılımında
meydana gelen dikdörtgen form yerini üst katlarda daha büyük kuvvetlerin
oluştuğu ters üçgen formuna bıraktığı görülmüştür.

Tez çalışması kapsamında elde edilen sonuçlar aşağıda listelenmiştir. 



\chapter{SONUÇLAR VE ÖNERİLER}

\section{Genel Değerlendirme}

Çalışmada elde edilen sonuçlara göre yalıtım birimlerinin artan eşdeğer
sönüm oranına karşılık gelen büyük doğrusalsızlık sebebi ile ortak
yalıtım düzleminde bulunan yapıların yüksek mod etkilerinin göreli
yer değiştirme ve iç kuvvetlerinde önemli artışlara sebep olduğu görülmüştür.
Bu artışların iki yapının açısal frekanslarının ayrıklaşması ile arttığı
tespit edilmiştir. Dolayısıyla iki yapının aynı açısal frekansa sahip
olması durumunda elde edilen iç kuvvet ve yer değiştirmeler, yalıtım
birimlerinin aynı eşdeğer periyot ve sönüm değerleri için eşit bulunmaktadır.
Bununla birlikte yalıtım birimlerinin artan eşdeğer sönüm değerleri
için sismik yalıtımlı yapı taban kesme kuvvetlerinin üst katlara dağılımında
meydana gelen dikdörtgen form yerini üst katlarda daha büyük kuvvetlerin
oluştuğu ters üçgen formuna bıraktığı görülmüştür.

Tez çalışması kapsamında elde edilen sonuçlar aşağıda listelenmiştir. 

Çalışmada elde edilen sonuçlara göre yalıtım birimlerinin artan eşdeğer
sönüm oranına karşılık gelen büyük doğrusalsızlık sebebi ile ortak
yalıtım düzleminde bulunan yapıların yüksek mod etkilerinin göreli
yer değiştirme ve iç kuvvetlerinde önemli artışlara sebep olduğu görülmüştür.
Bu artışların iki yapının açısal frekanslarının ayrıklaşması ile arttığı
tespit edilmiştir. Dolayısıyla iki yapının aynı açısal frekansa sahip
olması durumunda elde edilen iç kuvvet ve yer değiştirmeler, yalıtım
birimlerinin aynı eşdeğer periyot ve sönüm değerleri için eşit bulunmaktadır.
Bununla birlikte yalıtım birimlerinin artan eşdeğer sönüm değerleri
için sismik yalıtımlı yapı taban kesme kuvvetlerinin üst katlara dağılımında
meydana gelen dikdörtgen form yerini üst katlarda daha büyük kuvvetlerin
oluştuğu ters üçgen formuna bıraktığı görülmüştür.

Tez çalışması kapsamında elde edilen sonuçlar aşağıda listelenmiştir. 

\section{Gelecek Çalışmalara Yönelik Öneriler}

Çalışmada elde edilen sonuçlara göre yalıtım birimlerinin artan eşdeğer
sönüm oranına karşılık gelen büyük doğrusalsızlık sebebi ile ortak
yalıtım düzleminde bulunan yapıların yüksek mod etkilerinin göreli
yer değiştirme ve iç kuvvetlerinde önemli artışlara sebep olduğu görülmüştür.
Bu artışların iki yapının açısal frekanslarının ayrıklaşması ile arttığı
tespit edilmiştir. Dolayısıyla iki yapının aynı açısal frekansa sahip
olması durumunda elde edilen iç kuvvet ve yer değiştirmeler, yalıtım
birimlerinin aynı eşdeğer periyot ve sönüm değerleri için eşit bulunmaktadır.
Bununla birlikte yalıtım birimlerinin artan eşdeğer sönüm değerleri
için sismik yalıtımlı yapı taban kesme kuvvetlerinin üst katlara dağılımında
meydana gelen dikdörtgen form yerini üst katlarda daha büyük kuvvetlerin
oluştuğu ters üçgen formuna bıraktığı görülmüştür.

Tez çalışması kapsamında elde edilen sonuçlar aşağıda listelenmiştir. 


\bibliographystyle{itubib}
\bibliography{tez}

\noindent \eklerkapak{}

\vspace*{20pt}
\singlespacing

\noindent \textbf{EK A.1 :} Haritalar\textbf{}\\
\textbf{EK A.2 :} Diğer Haritalar\textbf{}\\
\textbf{EK A.3 :} Diğer Diğer Haritalar\\
\newpage{}

\noindent \eklerbolum{0}

\vspace*{20pt}


\chapter{EK A.1: Haritalar}

\begin{figure}[h!]
\centering{}\includegraphics[width=430pt]{fig/haritalar} \caption{\label{fig:6-1}Bölgesel haritalar: (a)Yağış. (b)Akım. (c)Evapotranspirasyon
...}
\end{figure}

\noindent 
\begin{table}
\caption{\label{tableappendix2}Ekler bölümünde çizelge örneği.}

\centering{}%
\begin{tabular}{cccc}
\hline 
Kolon A  & Kolon B  & Kolon C  & Kolon D \tabularnewline
\hline 
Satır A  & Satır A  & Satır A  & Satır A \tabularnewline
Satır B  & Satır B  & Satır B  & Satır B \tabularnewline
Satır C  & Satır C  & Satır C  & Satır C \tabularnewline
\hline 
\end{tabular}
\end{table}

Lorem ipsum dolor sit amet, consectetur adipiscing elit. Sed ac augue
vel dui adipiscing placerat et nec metus. Donec bibendum sodales mollis.
Cras in lacus justo, at vestibulum quam. Sed semper, est sit amet
consectetur ornare, leo est lacinia velit, adipiscing elementum lectus
felis at sem.

\newpage{}

\chapter{EK A.2: Diğer Haritalar}

Lorem ipsum dolor sit amet, consectetur adipiscing elit. Sed ac augue
vel dui adipiscing placerat et nec metus. Donec bibendum sodales mollis.
Cras in lacus justo, at vestibulum quam. Sed semper, est sit amet
consectetur ornare, leo est lacinia velit, adipiscing elementum lectus
felis at sem.

\noindent 
\begin{figure}[h!]
\centering{}\includegraphics[width=430pt]{fig/haritalar} \caption{\label{fig:6-1-1}Bölgesel haritalar: (a)Yağış. (b)Akım. (c)Evapotranspirasyon
...}
\end{figure}

\noindent 
\begin{table*}[!ht]
\caption{\label{tableappendix2-1}Ekler bölümünde çizelge örneği.}

\centering{}%
\begin{tabular}{cccc}
\hline 
Kolon A & Kolon B & Kolon C & Kolon D\tabularnewline
\hline 
Satır A & Satır A & Satır A & Satır A\tabularnewline
Satır B & Satır B & Satır B & Satır B\tabularnewline
Satır C & Satır C & Satır C & Satır C\tabularnewline
\hline 
\end{tabular}
\end{table*}

Lorem ipsum dolor sit amet, consectetur adipiscing elit. Sed ac augue
vel dui adipiscing placerat et nec metus. Donec bibendum sodales mollis.
Cras in lacus justo, at vestibulum quam. Sed semper, est sit amet
consectetur ornare, leo est lacinia velit, adipiscing elementum lectus
felis at sem.

\newpage{}

\chapter{EK A.3: Diğer Diğer Haritalar}

Lorem ipsum dolor sit amet, consectetur adipiscing elit. Sed ac augue
vel dui adipiscing placerat et nec metus. Donec bibendum sodales mollis.
Cras in lacus justo, at vestibulum quam. Sed semper, est sit amet
consectetur ornare, leo est lacinia velit, adipiscing elementum lectus
felis at sem.

\noindent 
\begin{figure}[h!]
\centering{}\includegraphics[width=430pt]{fig/haritalar} \caption{\label{fig:6-1-1-1}Bölgesel haritalar: (a)Yağış. (b)Akım. (c)Evapotranspirasyon
...}
\end{figure}

\noindent 
\begin{table*}[!ht]
\caption{\label{tableappendix2-1-1}Ekler bölümünde çizelge örneği.}

\centering{}%
\begin{tabular}{cccc}
\hline 
Kolon A & Kolon B & Kolon C & Kolon D\tabularnewline
\hline 
Satır A & Satır A & Satır A & Satır A\tabularnewline
Satır B & Satır B & Satır B & Satır B\tabularnewline
Satır C & Satır C & Satır C & Satır C\tabularnewline
\hline 
\end{tabular}
\end{table*}

Lorem ipsum dolor sit amet, consectetur adipiscing elit. Sed ac augue
vel dui adipiscing placerat et nec metus. Donec bibendum sodales mollis.
Cras in lacus justo, at vestibulum quam. Sed semper, est sit amet
consectetur ornare, leo est lacinia velit, adipiscing elementum lectus
felis at sem.

\newpage


\noindent \ozgecmis{\vspace{10mm}

\noindent \newsavebox{\mysquare}
\savebox{\mysquare}{\textcolor{black}{\rule[2.3pt]{3.4pt}{3.4pt}}}

\setlength{\TPHorizModule}{10pt}
\setlength{\TPVertModule}{10pt}

\begin{textblock}{1}(40,10)
 	\begin{figure}[p]
        % Resim dosyasını burada tanımlayın. Fotoğraf kullanımı zorunlu değildir.
		\includegraphics[width=3.5cm,keepaspectratio=true]{./fig/CV.jpg}
	\end{figure}
\end{textblock}

\noindent \textbf{Ad Soyad:} Ahmet Karabacak \\

\vspace{-3mm}
 \textbf{Doğum Tarihi ve Yeri:} 1993, Konya \\

\vspace{-3mm}
 \textbf{E-Posta:} ahmet7k@gmail.com \\

\textbf{ÖĞRENİM DURUMU:} \vspace{-3mm}
\begin{itemize}
	\item \textbf{Lise:} 2011, Konya Atatürk Lisesi 
	\item \textbf{Lisans:} 2016, Eskişehir Osmangazi Üniversitesi, İnşaat Mühendisliği
	\item \textbf{Y.Lisans:} 2019, İstanbul Teknik Üniversitesi, Deprem Mühendisliği 

\end{itemize}

\textbf{MESLEKİ DENEYİMLER VE ÖDÜLLER:} \vspace{-3mm}
\begin{itemize}
	\item 2013-2016 Soyut İnşaat, Yarı-zamanlı mühendislik uygulamaları
	\item 2014-2016 tarihlerinde Eskişehir Osmangazi Üniversitesi İnşaat Mühendisliği Bölüm Temsilciliği
	\item 2014 yılında TAV İnşaat Emaar Square projesinde saha stajı
	\item 2015 yılında Alarko Holding merkez ofis, arazi değerlendirme departmanında ofis stajı
	\item 2016- Halen, Erdemli Proje, Yapı Tasarım Mühendisi
	\item Bu tez 42127 numaralı İTÜ Bilimsel Araştırma Projesi kapsamında kabul edilmiştir.
\end{itemize}


% ---------------------------------------------------------------- %
% Fotografli ve yayin listeli (yayini varsa) ozgecmis onerilir.    %
% Fotograf ve adres sart degildir.                                 %
% -----------------------------------------------------------------%
}
\end{document}
