
\chapter{SONUÇLAR VE ÖNERİLER}

\section{Genel Değerlendirme}

Çalışmada elde edilen sonuçlara göre yalıtım birimlerinin artan eşdeğer
sönüm oranına karşılık gelen büyük doğrusalsızlık sebebi ile ortak
yalıtım düzleminde bulunan yapıların yüksek mod etkilerinin göreli
yer değiştirme ve iç kuvvetlerinde önemli artışlara sebep olduğu görülmüştür.
Bu artışların iki yapının açısal frekanslarının ayrıklaşması ile arttığı
tespit edilmiştir. Dolayısıyla iki yapının aynı açısal frekansa sahip
olması durumunda elde edilen iç kuvvet ve yer değiştirmeler, yalıtım
birimlerinin aynı eşdeğer periyot ve sönüm değerleri için eşit bulunmaktadır.
Bununla birlikte yalıtım birimlerinin artan eşdeğer sönüm değerleri
için sismik yalıtımlı yapı taban kesme kuvvetlerinin üst katlara dağılımında
meydana gelen dikdörtgen form yerini üst katlarda daha büyük kuvvetlerin
oluştuğu ters üçgen formuna bıraktığı görülmüştür.

Tez çalışması kapsamında elde edilen sonuçlar aşağıda listelenmiştir. 

Çalışmada elde edilen sonuçlara göre yalıtım birimlerinin artan eşdeğer
sönüm oranına karşılık gelen büyük doğrusalsızlık sebebi ile ortak
yalıtım düzleminde bulunan yapıların yüksek mod etkilerinin göreli
yer değiştirme ve iç kuvvetlerinde önemli artışlara sebep olduğu görülmüştür.
Bu artışların iki yapının açısal frekanslarının ayrıklaşması ile arttığı
tespit edilmiştir. Dolayısıyla iki yapının aynı açısal frekansa sahip
olması durumunda elde edilen iç kuvvet ve yer değiştirmeler, yalıtım
birimlerinin aynı eşdeğer periyot ve sönüm değerleri için eşit bulunmaktadır.
Bununla birlikte yalıtım birimlerinin artan eşdeğer sönüm değerleri
için sismik yalıtımlı yapı taban kesme kuvvetlerinin üst katlara dağılımında
meydana gelen dikdörtgen form yerini üst katlarda daha büyük kuvvetlerin
oluştuğu ters üçgen formuna bıraktığı görülmüştür.

Tez çalışması kapsamında elde edilen sonuçlar aşağıda listelenmiştir. 

\section{Gelecek Çalışmalara Yönelik Öneriler}

Çalışmada elde edilen sonuçlara göre yalıtım birimlerinin artan eşdeğer
sönüm oranına karşılık gelen büyük doğrusalsızlık sebebi ile ortak
yalıtım düzleminde bulunan yapıların yüksek mod etkilerinin göreli
yer değiştirme ve iç kuvvetlerinde önemli artışlara sebep olduğu görülmüştür.
Bu artışların iki yapının açısal frekanslarının ayrıklaşması ile arttığı
tespit edilmiştir. Dolayısıyla iki yapının aynı açısal frekansa sahip
olması durumunda elde edilen iç kuvvet ve yer değiştirmeler, yalıtım
birimlerinin aynı eşdeğer periyot ve sönüm değerleri için eşit bulunmaktadır.
Bununla birlikte yalıtım birimlerinin artan eşdeğer sönüm değerleri
için sismik yalıtımlı yapı taban kesme kuvvetlerinin üst katlara dağılımında
meydana gelen dikdörtgen form yerini üst katlarda daha büyük kuvvetlerin
oluştuğu ters üçgen formuna bıraktığı görülmüştür.

Tez çalışması kapsamında elde edilen sonuçlar aşağıda listelenmiştir. 
