Depremselliğin yüksek olduğu bölgelerde bulunan yapıların kuvvetli
yer hareketleri sebebiyle hasar almalarını engellemek amacıyla sismik
yalıtım sistemleri sıklıkla kullanılmaktadır. Sismik yalıtımlı yapılar
üst yapı, yalıtım düzlemi ve yalıtım birimlerinden oluşmaktadır. Bu
sistemler çoğunlukla izolatörlerin üzerinde göreli olarak rijit hareket
ettiği kabul edilen konut, veri merkezi binası, hastane yapıları,
sıvı tankı ve açık deniz petrol platformu gibi tek yapılardır. Birbirinden
bağımsız sismik yalıtımlı yapıların birlikte kullanılmaları durumunda
çözülmesi gereken mimari detaylar karmaşık ve pahalı olmaktadır. Bu
sebeple pratikte birbirinden bağımsız bina tipi yapılar ortak yalıtım
düzleminde tasarlanmaktadır. Özellikle Türkiye'de son yıllarda yapılan
hastane yapıları bu tanıma uymaktadır.

Geçmişte yapılan çalışmalar neticesinde belirlenen doğrusal tasarım
yöntemleri, sismik yalıtımlı yapıları tek veya çok serbestlik dereceli
bir yapıdan oluşan sistemler olarak değerlendirmektedir. Bu nedenle
yönetmeliklerde yer alan tasarım prosedürleri bağımsız yalıtım düzleminde
bulunan sismik yapılar hazırlanmıştır. Ortak yalıtım düzleminde bulunan
sismik yalıtımlı yapılar için ise yönetmeliklerde tasarım kriteri
bulunmamaktadır. Pratikte ortak yalıtım düzleminde tasarlanmış yapıların
her biri bağımsız yalıtım düzleminde analiz edilerek boyutlandırılmaktadır.
Bu sebeple izolatörlerin histeretik davranışlarından kaynaklanan yüksek
mod etkileri dolayısıyla yapıların dinamik etkileşimi değerlendirilememektedir.

Bu çalışmanın amacı ortak yalıtım düzleminde bulunan iki yapı için
kapsamlı parametrik inceleme yapılarak üst yapıların dinamik etkileşimi
sebebiyle oluşan taban kesme kuvvetlerindeki amplifikasyonun irdelenmesidir.

Çalışma kapsamında ortak yalıtım düzlemine sahip iki yapının dinamik
etkileşimi, parametrik olarak değişen sistemler üzerinden incelenmiştir.
Ayrıca ortak yalıtım düzleminde bulunan üç ve dört yapılı sistemler
için örnek çözümler yapılmıştır. Burada üst yapı kat sayısı, yalıtım
birimleri eşdeğer periyot ve eşdeğer sönüm değerleri birer parametre
olarak ele alınmıştır. Ortak yalıtım düzleminde bulunan iki yapı kat
sayısının birden ona değiştiği durumlar için analizler yapılmıştır.
Tüm analizler yalıtım birimlerinin 1.5, 2.5 ve 4.0sn eşdeğer periyot
ve \%10, \%20 ve \%30 eşdeğer sönüm değerleri için tekrarlanmıştır.
Yalıtım birimlerinin histeretik özellikleri, değişen üst yapı kütlesine
karşılık doğrusal olmayan analizler sonucunda sabit eşdeğer periyot
ve eşdeğer sönüm değerlerini elde edebilmek amacıyla normalize edilmiştir.
Bu işlem istenilen eşdeğer periyot değerine karşılık gelen rijitlik
ve eşdeğer sönüm değerleri ile yapılan doğrusal zaman tanım analizlerinden
elde edilen yalıtım birimi yer değiştirmelerine karşılık akma sonrası
rijitlik ve karakteristik dayanımın belirlenmesi ile gerçekleştirilmektedir.

Analizler yalnızca yapıların ortak yalıtım düzleminde bulunduğu doğrultu
için yapılmıştır. Diğer yöndeki etkiler ve yalıtım birimlerinin iki
yönlü etkileşimi göz ardı edilmiştir. Üst yapıların kat kütle ve kat
rijitlikleri tüm analizler için sabit alınmıştır. Yalıtım düzlemi
kütlesi ise her yapı için sabit olup ortak yalıtım düzleminde bulunan
yapı sayısı çarpılarak hesaplanmıştır. Yalnızca kesme yaylarının dikkate
alındığı modellerde üst yapılar çok serbestlik dereceli ve doğrusal
elastik olarak tanımlanmıştır. Yalıtım birimlerinin doğrusal olmayan
davranışları ise çift doğrulu eleman ile temsil edilmektedir. Yalıtım
birimlerinde yalnızca doğrusal olmayan davranıştan kaynaklanan histeretik
sönüm kullanılmıştır. Analizlerde kullanılan deprem kayıtları, 50
yılda aşılma olasılığı \%2 olan deprem yer hareketi düzeyine göre
oluşturulan tasarım spektrumuna uygun olarak eşleştirilmiştir.

Tüm analizler tez kapsamında geliştirilen, bağımsız ve ortak yalıtım
düzleminde bulunan sismik yalıtımlı yapıların doğrusal olmayan analizlerinin
parametrik olarak çözülebilmesine imkan sağlayan MSBIS programı yardımıyla
gerçekleştirilmiştir. Geliştirilen programın doğruluğu, genel kabul
görmüş yapısal analiz programı olan SAP2000 ile karşılaştırılarak
gösterilmiştir.

Yer ivmeleri etkisinde doğrusal olmayan dinamik analizler yapılarak
birinci yapıda meydana gelen kat kesme kuvvetleri, kat ivmeleri ve
göreli kat ötelemeleri ikinci yapının değişen açısal frekansına ve
yalıtım birimlerinin farklı eşdeğer periyot ve eşdeğer sönüm değerlerine
göre değişimi incelenmiştir. Ayrıca birinci yapının on katlı olması
durumunda ikinci yapının değişen kat sayıları için kat kesme kuvveti
katsayıları değişimi irdelenmiştir. Elde edilen sonuçlar, incelenen
yapıların bağımsız yalıtım düzleminde bulunması durumunda elde edilecek
büyüklüklere göre normalize edilmiştir. Böylece yapıların taban kesme
kuvvetlerindeki amplifikasyon üst yapı açısal frekanslarına, yalıtım
birimi eşdeğer periyot ve eşdeğer sönümüne bağlı olarak tespit edilebilmektedir.
Ortak yalıtım düzleminde bulunan yapıların taban kesme kuvveti katsayıları
elde edilerek sonuçlar yapıların dinamik analizde vektörel olarak
toplanarak elde edilen toplam kesme kuvveti katsayısına göre kıyaslanmıştır.
Tespit edilen bu hata oranı yapıların dinamik etkileşiminin mertebesi
ile doğru orantılı olarak artmaktadır. Buna ek olarak ortak yalıtım
düzleminde bulunan iki yapının gerekli deprem derz mesafeleri, yönetmeliklerde
belirtilen yöntemler ile hesaplanarak doğrusal olmayan dinamik analizlerden
elde edilen sonuçlar ile kıyaslanmıştır.

Çalışma sonucunda, yalıtım birimlerinin histeretik sönümlerinin artması
nedeniyle oluşan yüksek mod etkilerinin ortak yalıtım düzleminde bulunan
yapıların taban kesme kuvveti katsayılarını, göreli kat ötelemelerini
ve en büyük kat ivmelerini önemli ölçüde artırdığı tespit edilmiştir.
Bu artışın, iki yapının açısal frekansların ayrıklaşması ile arttığı
görülmüştür. Dolayısıyla iki yapının da aynı açısal frekansa sahip
olması durumunda elde edilen iç kuvvet ve yer değiştirmeler, yalıtım
birimlerinin aynı eşdeğer periyot ve sönüm değerleri için eşit bulunmaktadır.
Ortak yalıtım düzleminde bulunan iki yapıdan incelenen bir katlı yapının
taban kesme kuvvetleri ikinci yapının değişen kat sayılarına karşılık
bağımsız yalıtım düzleminde bulunması durumuna göre artmaktadır. On
katlı yapının incelendiği durumda ise kat kesme kuvvetleri alt katlarda
artarken üst katlar için azaldığı belirlenmiştir. Ortak yalıtım düzleminde
bulunan üç ve dört yapılı sistemler için yapılan analizler sonucunda
ise incelenen yapıdan farklı açısal frekansa sahip yapı sayısının
artmasıyla taban kesme kuvveti katsayılarının arttığı tespit edilmiştir.
Ayrıca histeretik sönümün artması ile taban kesme kuvvetinin üst yapıya
dağılımında ters üçgen formunun oluştuğu belirlenmiştir.
