	Moment dayanımlı çelik çerçeveli yapılar emniyet,ekonomi ve estetik
	gibi yapı tasarımının önemli performans parametreleri ile öne çıkmaktadır.
	Moment dayanımlı çelik çerçeveler yüksek süneklik kapasiteleri ile
	deprem bölgelerinde az ve orta katlı ofis türü binalarda sıklıkla
	kullanılmaktadır. Moment çerçevelerinin yüksek süneklik özellikleri
	yatay kuvvetlere maruz kaldığında sistemin enerji sönümm mekanizmasındaki
	basitlikte yatmaktadır. Bu sistemlerde özellikle deprem kuvvetleri
	altında yapının dayanımında azalma olmadan kirişlerde çok sayıda akma
	elastik olmayan şekil değiştirme ile sağlanır.
	
	Bir yapıdaki bütün kolon-kiriş birleşimlerinin tasarımda tam rijit
	veya ideal mafsallı kabulünün ne derece doğru olduğu tartışma konusudur.
	Gerçekte moment çerçevesinde kullanılan birleşimlerin davranışı incelendiğinde
	birleşimlerin rijit veya ideal mafsal davranışı göstermesi için çok
	özel detaylandırmaların kullanılması şarttır halbuki kullanılan birleşimlerin
	çoğunluğu rijit ve mafsallı iki uç sınır durumun arasında kalmaktadır.
	Bu birleşim türleri yarı-rijit birleşim olarak sınıflandırılmaktadır. 
	
	Yer değiştirme arttırma katsayısı, yapının elastik durumdaki yer değiştirme
	değerinden, elastik ötesi durumdaki yer değiştirme değerini elde etmek
	için kullanılmaktadır. 
