Moment dayanımlı çelik çerçeveli yapılar emniyet,ekonomi ve estetik
gibi yapı tasarımının önemli performans parametreleri ile öne çıkmaktadır.
Moment dayanımlı çelik çerçeveler yüksek süneklik kapasiteleri ile
deprem bölgelerinde az ve orta katlı ofis türü binalarda sıklıkla
kullanılmaktadır. Moment çerçevelerinin yüksek süneklik özellikleri
yatay kuvvetlere maruz kaldığında sistemin enerji sönümm mekanizmasındaki
basitlikte yatmaktadır. Bu sistemlerde özellikle deprem kuvvetleri
altında yapının dayanımında azalma olmadan kirişlerde çok sayıda akma,
elastik olmayan şekil değiştirme ile sağlanır. Bu tip yapılarda plastik
mafsalların kolonlardan önce kirişlerde oluşmasını sağlamak, yapının
sünekliğini arttırıp, göçmesini geciktirmek için güçlü kolon zayıf
kiriş ilkesi uygulanır. Bu kural gerekenden daha büyük kolon kesitlerinin
seçilmesine ve ekonomik olmayan, aşırı güvenli tasarımlarla sonuçlanmaktadır.
Bu sorunu gidermek için özellikle Amerika’da uygulanan yöntem yapıların
sadece dış çerçevelerini yatay yük taşıyan sistemler olarak tasarlamaktır.
Ancak bu yapıların başlıca dezavantajlarından biri, sistemin yeniden
dağılım imkânının sınırlı olmasıdır. 

1994 Northridge ve 1995 Kobe depremleri sırasında rijit birleşimli
moment dayanımlı çelik çerçeveli yapılarda ve özellikle tamamen kaynaklı
birleşimlerde ağır ve beklenilmeyen hasarlar oluşmuş ve bu tip yapıların
tasarım yöntemlerinin gözden geçirilmesine neden olmuştur. Bu keşiflerin
ardından, SAC Ortak Girişimi olarak bilinen profesyonel derneklerin
ve araştırmacılardan oluşan bir konsorsiyum, bu beklenmeyen davranışın
nedenlerini belirlemek ve daha sağlam moment çerçevesi inşası için
öneriler geliştirmek için kapsamlı bir çalışma yürütmüştür. Çatlakların
ve hasarların, temel birleşim geometrisi, temel malzemelerin kontrolünün
eksikliği, iç yapıdaki düşük tokluğa sahip kaynak dolgu metallerinin
kullanılması, kolon başlıklarında divot kırılması, karşılama levhası
kaynaklı hasarlar, yetersiz kalite kontrol ve diğer faktörlerin bir
sonucu olduğu tespit edildi. Ortaya çıkan araştırma, mevcut özel moment
çerçeveleri için yönetmelik tasarım gerekliliklerinin temelini oluşturmuştur.

Moment çerçeveleri büyük yer değiştirme yapabilme özellikleri hem
avantaj hemde dezavantajı olarak söylenebilir. Bir yapıdaki bütün
kolon-kiriş birleşimlerinin tasarımda tam rijit veya ideal mafsallı
kabulünün ne derece doğru olduğu tartışma konusudur. Gerçekte moment
çerçevesinde kullanılan birleşimlerin güçlü veya zayıf eksende olsun,
davranışı incelendiğinde birleşimlerin rijit veya ideal mafsal davranışı
göstermesi için çok özel detaylandırmaların kullanılması şarttır halbuki
kullanılan birleşimlerin çoğunluğu rijit ve mafsallı iki uç sınır
durumun arasında kalmaktadır. Bu birleşim türleri yarı-rijit birleşim
olarak sınıflandırılmaktadır. 

Geçtiğimiz yarım asırlık süreçte yapısal tasarım için mühendislerin
kullandığı bilgisayar destekli yazılım araçları önemli gelişme kaydetmiştir.
Ancak hala çelik yapıların modellenmesi ve tasarımı aşamasında yapılan
bazı kabuller göze çarpmaktadır. Özellikle deprem bölgesinde tasarlanan
bir çelik yapı için güvenlik ve ekonomi amacı büyük önem arz eder.
Deprem yönetmeliklerinde yer alan yapısal tasarım felsefesi, makul
bir oluşma olasılığı olan bir depremde bile mutlak güvenlik ve hasarsızlığın
sağlanamayacağının altını çizmektedir. Bununla birlikte, bazı yapısal
ve yapısal olmayan hasarlara izin verip elastik olmayan enerjiyi sönümleyerek
yapılarda yüksek bir yaşam güvenliği seviyesini ekonomik olarak sağlamayı
hedefler. Bu tasarım felsefesinin bir sonucu olarak, yönetmeliklerde
öngörülen yatay tasarım kuvveti yapıyı elastik aralıkta tutmak için
gereken yatay kuvvetten daha düşüktür. Yapının elastik zarfta tutulması,
yanal harekete maruz kalan tüm yapısal elemanların, herhangi bir kalıcı
deformasyon ve hasar almadan başlangıç durumuna geri dönmesinin garanti
edilmesi anlamına gelir ki bu durumu korumak, uygulanabilir ve rasyonel
olmaktan uzaktır. 

Moment çerçevelerinin tasarımı göz önüne alındığında tasarımı yöneten
ana parametrenin yer değiştirme olduğu ortaya çıkar. Bu çalışmada
literatürde ve deprem yönetmeliklerinde moment dayanımlı çelik çerçeveli
yapıların tasarımında birleşim dönme rijitliği kavramı ve yönetmeliklerin
tasarımda kullanılacak gerçek yer değiştirme için tepki değişitirme
katsayısı kavramı üzerinde durmaktadır. Yer değiştirme arttırma katsayısı,
yapının elastik durumdaki yer değiştirme değerinden, elastik ötesi
durumdaki yer değiştirme değerini elde etmek için kullanılmaktadır.
Bu amaç doğrultusunda az,orta ve yüksek katlı rijit birleşimli moment
çerçeveli yapılar belirlenen deprem düzeyi için tasarlanmış, beş farklı
dönme rijitliği ile doğrusal elastik olmayan statik ve dinamik analizlerle
incelenmiştir. 15 adet çelik çerçeve statik itme analizleri ve tasarım
depremi seviyesinde ölçeklendirilmiş 18 adet gerçek yer hareketi ile
dinamik analizler yapılmıştır. Bu analiz sonuçları ile her bir çelik
çerçeve için yer değiştirme arttırma katsayısı hesaplanmıştır.

Az, orta ve yüksek katlı çerçevelerde, beş farklı birleşim rijitliği
altında, statik analizler sonucu, ortalama 4.84, dinamik analizler
sonucu, ortalama 3.86 olarak hesaplanan $C_{\mathrm{d}}$ katsayıları
AISC ve TBDY-2018 yönetmeliklerinin sırasıyla kullandığı 5.50 ve 8.00
değerinden daha düşüktür. Sonuçlar çelik çerçeveli bir yapıda birleşim
rijitliğinin yapının davranışını değiştirdiğini, dolayısı ile çelik
çerçeveli yapılar için önemli bir tepki değiştrme katsayısı olan yer
değiştirme arttırma katsayısının değiştirdiği anlaşılmaktadır. Bulunan
yer değiştirme arttırma katsayılarındaki farklılık, moment dayanımlı
bir çelik çerçevenin doğrusal olmayan değerlendirmesi yapılması durumunda
yapıda kullanılan birleşimlerin rijitliklerinin göz ardı edilmemesi
gerektiğini göstermektedir. Ayrıca birleşim rijitliğinin azalmasının
yapıların süneklik oranının azaldığını göstermiştir.